\documentclass[lang=cn, thmcnt=section, chinesefont=founder, color=cyan, citestyle=gb7714-2015, bibstyle=gb7714-2015]{elegantbook}

\usepackage[warnings-off={mathtools-colon,mathtools-overbracket}]{unicode-math}
\setmathfont{STIXTwoMath-Regular.otf}

% \newcommand{\symbf}[1]{\mathbf{#1}}
% \newcommand{\symrm}[1]{\mathrm{#1}}
% \newcommand{\symcal}[1]{\mathcal{#1}}

\usepackage{minted}
\usepackage[color=black]{siunitx}
\usepackage[ISO]{diffcoeff}
\usepackage{tikz}
\usepackage{tkz-euclide}
\usepackage{tkz-graph}
\usepackage{dashrule}
\usepackage{float}
\usepackage{fontawesome}
\usepackage{multirow}
\usepackage{longtable}

\ctexset{
    punct=kaiming
}

\newcommand{\qed}{\hfill$\QED$}

\PassOptionsToPackage{dvipsnames,svgnames,x11names}{xcolor}

\definecolor{shadow}{RGB}{210,241,241}
\definecolor{information}{RGB}{220,234,247}
\definecolor{infotitle}{RGB}{34,64,113}

\titleformat{\part}[display]{\vspace*{\fill}\bfseries}{
  \hspace*{\fill}\zihao{1}\enspace\bfseries{\color{structurecolor} \thepart}}{0pt}{
  \hspace*{\fill}\zihao{-0}\bfseries\color{black}}[\vspace{-1em}\color{structurecolor}\rule{\textwidth}{2pt}\vspace*{\fill}\newpage]
% \titleformat{\section}[hang]{\vspace*{0.5em}\bfseries}{
%   \filcenter\LARGE\bfseries{\color{structurecolor}\thesection}\enspace}{0pt}{%
%   \color{structurecolor}\LARGE\bfseries\filcenter}[\vspace{-1.3em}\hspace*{\fill}\color{structurecolor}\rule{0.2\textwidth}{1pt}\hspace*{\fill}\vspace*{1.3em}]
% \titleformat{\subsection}[hang]{\bfseries}{
%   \Large\bfseries\color{structurecolor}\thesubsection\enspace}{1pt}{%
%   \color{structurecolor}\large\bfseries\filright}
% \titleformat{\subsubsection}[hang]{\bfseries}{
%   \large\bfseries\color{structurecolor}\thesubsubsection\enspace}{1pt}{%
%   \color{structurecolor}\large\bfseries\filright}

\tcbuselibrary{minted}
\newcommand{\spare}{\vspace{-1em}\begin{center}\color{structurecolor}\hdashrule[0.5ex]{\textwidth}{1pt}{1pt}\end{center}\vspace{-1em}}
\usetikzlibrary{shapes,backgrounds}
\ExecuteBibliographyOptions{sorting=gb7714-2015}
\setlength{\parskip}{1ex}
\newtcolorbox{collections}{
      boxrule=0.5pt,
      enhanced,
      breakable,
      top=8pt,
      before skip=8pt,
      colframe=structurecolor,
      colback=structurecolor!5,
      colbacktitle=structurecolor
}
\newtcolorbox{info}[1]{
      boxrule=0.5pt,
      enhanced,
      breakable,
      top=8pt,
      before skip=8pt,
      title = \vspace*{7pt}\color{infotitle}\faInfoCircle\enspace\textbf{#1},
      colframe=information,
      colback=information,
      colbacktitle=information
}
\newtcblisting{code*}[1]{
    listing engine=minted,
    boxrule=0.1mm,
    colback=third!5,
    colframe=third,
    listing only,
    left=5mm,
    enhanced,
    breakable,
    overlay={\begin{tcbclipinterior}\fill[third] (frame.south west)
    rectangle ([xshift=5mm]frame.north west);\end{tcbclipinterior}},
    minted language=#1,
    minted style=vs,
    minted options={breaklines, mathescape, autogobble, linenos, numbersep=2mm}
}
\newtcblisting[auto counter]{code}[2]{
    listing engine=minted,
    boxrule=0.1mm,
    colback=third!5,
    colframe=third,
    listing only,
    left=5mm,
    enhanced,
    breakable,
    title=Program \Alph{\tcbcounter}: #2,
    overlay={\begin{tcbclipinterior}\fill[third] (frame.south west)
    rectangle ([xshift=5mm]frame.north west);\end{tcbclipinterior}},
    minted language=#1,
    minted style=vs,
    minted options={breaklines, mathescape, autogobble, linenos, numbersep=2mm}
}
\renewcommand{\theFancyVerbLine}{\textcolor{white}{\footnotesize{\arabic{FancyVerbLine}}}}

\renewcommand{\bar}[1]{\overline{#1}}
\renewcommand{\emph}[1]{\textit{\bfseries #1}}
\arrayrulecolor{second}

\usetikzlibrary{graphs}

\title{新世代计算机科学计划·基础篇}
\author{JouderMin}
\institute{「新世代计算机科学计划」制作委员会}
\date{\zhtoday}
\cover{img/cover.png}
\logo{img/方形logo.png}

\begin{document}
\maketitle
\frontmatter

\tableofcontents

\mainmatter

\chapter{离散数学}
\section{集合}
\begin{introduction}
    \item 集合与元素
    \item 集合的表示
    \item 子集
    \item 幂集
    \item 有序 $n$ 元组与序偶
    \item 笛卡尔积
\end{introduction}

\subsection{集合与元素}
离散数学的多数内容主要研究用以表示离散对象的离散结构,而集合则是最基础的离散结构,在本节内容中我们将了解什么是集合,以及一些与集合有关的基础概念。

以下是集合与元素的一个定义。
\begin{definition}[集合与元素]\label{def:集合与元素}
    集合是不同对象的无序聚集,这些对象也被称为集合的元素或成员。集合包含他的元素。$a \in A$ 表示 $a$ 是集合 $A$ 中的一个元素,$a \notin A$ 表示 $a$ 不是集合 $A$ 中的一个元素。
\end{definition}

我们常用大写字母表示集合,用小写字母表示元素。
\begin{definition}[相等的集合]\label{def:相等的集合}
    设有集合 $A$ 与 集合 $B$,当且仅当他们拥有相同元素时集合 $A$ 与集合 $B$ 相等,记作$A = B$。
\end{definition}

\begin{collections}
    \begin{example}
        判断下列集合是否相等。
        \begin{enumerate}
            \item $A = \{ 1, 4, 5 \}$,$B = \{ 1, 5 ,4 \}$
            \item $C = \{ \{ 1, 4 \}, 5 \}$,$D = \{ \{ 1, 5 \}, 4 \}$
            \item $E = \{ \{ 1, 4, 5 \} \}$,$F = \{ \{ 1, 5, 4 \} \}$
        \end{enumerate}
    \end{example}
    \begin{solution}
        集合 $A$ 与集合 $B$ 拥有相同的元素,所以 $A = B$;集合 $C$ 与集合 $D$ 没有相同的元素,所以 $C \neq D$;集合 $E$ 与集合 $F$ 拥有相同的元素,所以 $E = F$。
    \end{solution}
\end{collections}

集合中元素的数量被称为集合的基数,以下为基数的数学定义
\begin{definition}[集合的基数]\label{def:集合的基数}
    令 $S$ 为集合,如果集合 $S$ 中恰有 $n$ 个不同的元素($n$ 为非负整数),则称 $S$ 为有限集,$n$ 为集合 $S$ 的基数,记为 $|S|$。
\end{definition}

只有一个元素的集合叫做单元素集,单元素集的基数为 $1$;不含任何元素的集合被称为空集,记为 $\varnothing$,空集的基数为 $0$。
\begin{collections}
    \begin{example}
        判断以下集合是否空集或单元素集:
        \begin{enumerate}
            \item $A = \varnothing$
            \item $B = \{ 1 \}$
            \item $C = \{ 1, 2 \}$
            \item $D = \{ \varnothing \}$
            \item $E = \{\{ 1, 2 \}\}$
        \end{enumerate}
    \end{example}
    \begin{solution}
        集合 $A$ 是空集,因为 $|A| = 0$;集合 $B$ 是单元素集,因为 $|B| = 1$;集合 $C$ 既不是空集也不是单元素集,因为 $|C| = 2$;集合 $D$ 是单元素集,因为 $|D| = 1$;集合 $E$ 是单元素集,因为 $|E| = 1$。
    \end{solution}
\end{collections}


\subsection{集合的表示}
在离散数学中有一些常用的集合,一般用黑体大写字母表示。
\begin{itemize}
    \item $\symbf{N}$ 为所有自然数的集合;
    \item $\symbf{Z}$ 为所有整数的集合;
    \item $\symbf{Z}^+$ 为所有正整数的集合;
    \item $\symbf{Q}$ 为所有有理数的集合;
    \item $\symbf{R}$ 为所有实数的集合;
    \item $\symbf{C}$ 为所有复数的集合。
\end{itemize}

描述集合有多种方式,人们常用以下三种方式来表示集合:
\begin{itemize}
    \item 花名册方法(俗称枚举法);
    \item 集合构造器符号(俗称描述法);
    \item 文氏图。
\end{itemize}

花名册方法,俗称枚举法,这种方法需要在可能的情况下一一列出集合中的元素。在集合中关系显然的情况下,可以用 $\cdots$ 省略部分元素。
\begin{collections}
    \begin{example}
        使用花名册方法表示小于 $10$ 的正素数集合 $O$ 。
    \end{example}
    \begin{solution}
        $O = \{ 2, 3, 5, 7 \}$
    \end{solution}

    \spare

    \begin{example}
        使用花名册方法表示小于等于 $100$ 的所有自然数集合 $Q$。
    \end{example}
    \begin{solution}
        $Q = \{ 0, 1, 2, \cdots, 100 \}$
    \end{solution}
\end{collections}

我们也可以使用集合构造器符号来表示集合,这种方法俗称描述法,需要描述集合中元素必须具有的性质。一般采用记号 $\{x \mid x\text{\,具有性质\,}P\}$。
\begin{collections}
    \begin{example}
        使用集合构造器符号表示大于等于 $10$,小于 $100$ 的正整数集合 $P$ 。
    \end{example}
    \begin{solution}
        $P=\{x \in Z^+ \mid 10 \leq x < 100\}$
    \end{solution}
\end{collections}

集合也可以用文氏图来表示。在文氏图中,全集 $U$ 包含所考虑的所有元素,用矩形框表示,矩形框中的圆形或其他几何形状用于表示集合,点表示集合中的元素。文氏图常用来表示集合之间的关系。
\begin{collections}
    \begin{example}
        设集合 $A$ 包含在全集 $U$ 中,用文氏图表示集合 $A$ 与全集 $U$。
    \end{example}
    \begin{solution}
        \begin{center}
            \begin{tikzpicture}
                \draw[thick] (0,0) rectangle (4,3);
                \draw[thick] (2,1.5) circle (1);

                \node[below left] at (4,3) {$U$};
                \node at (2,1.5) {$A$};
            \end{tikzpicture}
        \end{center}
    \end{solution}
\end{collections}

\subsection{子集}
子集是集合之间的一种关系。
\begin{definition}[子集与超集]\label{def:子集与超集}
    当且仅当集合 $A$ 中的所有元素都是集合 $B$ 的元素时,集合 $A$ 称为集合 $B$ 的子集,记作 $A \subseteq B$,集合 $B$ 称为集合 $A$ 的超集,记作 $B \supseteq A$。
\end{definition}

在文氏图中可以直观表示子集关系。设集合 $B$ 是集合 $A$ 的子集,集合 $U$ 是全集,则可以作出图 \ref{fig:子集文氏图}。
\begin{figure}[htbp!]
    \centering
    \begin{tikzpicture}
        \draw[thick] (0,0) rectangle (4,3);
        \draw[thick] (2,1.5) circle (1.2);
        \draw[thick] (2,1.5) circle (0.5);

        \node[below left] at (4,3) {$U$};
        \node at (2,1.5) {$B$};
        \node at (2.8,1.5) {$A$};
    \end{tikzpicture}
    \caption{表示 $B \subseteq A$ 的文氏图}
    \label{fig:子集文氏图}
\end{figure}

子集关系有以下定理:
\begin{theorem}\label{thm:集合与空集之间的关系}
    对任意集合 $S$,总有 $\varnothing \subseteq S$。
\end{theorem}

\begin{theorem}\label{thm:集合与其自身的关系}
    对任意集合 $S$,总有 $S \subseteq S$。
\end{theorem}

定理 \ref{thm:集合与其自身的关系} 可以用于证明两个集合是否相等。
\begin{theorem}\label{thm:证明集合相等}
    对于集合 $A$,如果存在集合 $B$ 使 $A \subseteq B$ 与 $B \subseteq A$ 成立,则 $A = B$。
\end{theorem}

\subsection{幂集}
\begin{definition}[幂集的定义]\label{def:幂集的定义}
    给定集合 $S$,$S$ 的幂集是集合 $S$ 所有子集的集合,$S$ 的幂集记为 $\symcal{P}(S)$ 或 $2^S$。
\end{definition}

\begin{collections}
    \begin{example}
        求以下集合的幂集。
        \begin{enumerate}
            \item $A = \{1, 2, 3\}$
            \item $B = \{ \varnothing \}$
            \item $C = \{\{ \varnothing \}\}$
        \end{enumerate}
    \end{example}
    \begin{solution}
        \begin{enumerate}
            \item $\symcal{P}(A) = \{\varnothing, 1, 2, 3, \{1, 2\}, \{1, 3\}, \{2, 3\}, \{1, 2, 3\}\}$
            \item $\symcal{P}(B) = \{\varnothing\}$
            \item $\symcal{P}(C) = \{\varnothing, \{\varnothing\}\}$
        \end{enumerate}
    \end{solution}
\end{collections}

\begin{theorem}
    令 $S$ 为集合,设 $|S| = n$,则$|\symcal{P}(S)| = 2^n$。
\end{theorem}

\subsection{有序 $n$ 元组与序偶}
集合中的元素是无序排列的,所以我们需要一种结构来表示有序的聚集,这便是有序 $n$ 元组。
\begin{definition}[有序 $n$ 元组与序偶]\label{def:有序n元组与序偶}
    有序 $n$ 元组 $(a_1, a_2, \cdots, a_n)$ 是以 $a_1$ 为第 $1$ 个元素,$a_2$ 为第 $2$ 个元素,$\cdots$,$a_n$ 为第 $n$ 个元素的有序聚集,有序 $n$ 元组还可以被称作序列。特别的,有序二元组被称为序偶。
\end{definition}

两个有序 $n$ 元组相等当且仅当每一对对应的元素相等。
\begin{collections}
    \begin{example}
        判断 $A=(1, 2)$ 与 $B=(2, 1)$ 是否相等。
    \end{example}
    \begin{solution}
        $A \neq B$
    \end{solution}
\end{collections}

\subsection{笛卡尔积}
\begin{definition}[笛卡尔积]\label{def:笛卡尔积}
    令 $A$ 和 $B$ 为集合,$A$ 与 $B$ 的笛卡尔积(记作 $A \times B$)是所有序偶 $(a,b)$ 的集合,其中 $a \in A$ 且 $b \in B$。
    \begin{equation*}
        A \times B = \{(a,b) \mid a \in A \land b \in B \}
    \end{equation*}
\end{definition}

\begin{collections}
    \begin{example}
        已知$A = \{1, 2\}$,$B = \{a, b, c\}$,求 $A \times B$ 和 $B \times A$。
    \end{example}
    \begin{solution}
        $$A \times B = \{(1, a), (1, b), (1, c), (2, a), (2, b), (2, c)\}$$
        $$B \times A = \{(a, 1), (a, 2), (b, 1), (b, 2), (c, 1), (c, 2)\}$$
    \end{solution}
\end{collections}

\newpage

\section{集合的运算}
\begin{introduction}
    \item 交集
    \item 并集
    \item 差集
    \item 补集
    \item 对称差
\end{introduction}

\subsection{交集}
\begin{definition}[集合的交集]\label{def:交集}
    令 $A$ 与 $B$ 为集合。集合 $A$ 与 $B$ 的交集是一个集合,它包含集合 $A$ 与 $B$ 中共有的元素,记作 $A \cap B$。
    \begin{equation*}
        A \cap B = \{x \mid x \in A \land x \in B \}
    \end{equation*}
\end{definition}

集合 $A$ 与 $B$ 的交集可以用图 \ref{fig:交集文氏图} 中的阴影部分表示。
\begin{figure}[htbp!]
    \centering
    \begin{tikzpicture}[scale=0.75]
        \begin{scope}
            \clip (2,2) circle (1.5);
            \fill[shadow] (4,2) circle (1.5);
        \end{scope}

        \draw[thick] (-0.5,0) rectangle (6.5,4);
        \draw[thick] (2,2) circle (1.5);
        \draw[thick] (4,2) circle (1.5);

        \node[below left] at (6.5,4) {$U$};
        \node at (1.75,2) {$B$};
        \node at (4.25,2) {$A$};
    \end{tikzpicture}
    \caption{$A \cap B$ 的文氏图}
    \label{fig:交集文氏图}
\end{figure}

由此我们还可以定义一组集合的交集。
\begin{definition}[多个集合的交集]\label{def:多个交集}
    一组集合的交集是包含这组集合中所有成员集合共有的元素的集合。
\end{definition}

我们用符号
\begin{equation*}
    \bigcap_{i=1}^n A_i=A_1 \cap A_2 \cap \cdots \cap A_i
\end{equation*}
表示 $A_1$,$A_2$,$\cdots$,$A_n$ 的交集。
\begin{collections}
    \begin{example}
        已知 $A = \{ 1, 2, 3 \}$,$B = \{ 2, 3, 4\}$,求 $A \cap B$。
    \end{example}
    \begin{solution}
        $A \cap B = \{2, 3\}$
    \end{solution}

    \spare

    \begin{example}
        已知 $A = \{ x \mid 10 < x \leq 20 \}$,$B = \{ x \mid 0 < x \leq 10 \}$,求 $A \cap B$。
    \end{example}
    \begin{solution}
        $A \cap B = \varnothing$
    \end{solution}

    \spare

    \begin{example}
        已知 $A = \{ x \mid 10 \leq x \leq 20 \}$,$B = \{ x \mid 0 \leq x \leq 10 \}$,求 $A \cap B$。
    \end{example}
    \begin{solution}
        $A \cap B = \{10\}$
    \end{solution}
\end{collections}

\subsection{并集}
\begin{definition}[集合的并集]\label{def:并集}
    令 $A$ 与 $B$ 为集合。集合 $A$ 与 $B$ 的并集是一个集合,它包含 $A$ 和 $B$ 中的所有元素,记作 $A \cup B$。
    \begin{equation*}
        A \cup B = \{x \mid x \in A \lor x \in B\}
    \end{equation*}
\end{definition}

集合 $A$ 与 $B$ 的并集可以用图 \ref{fig:并集文氏图} 中的阴影部分表示。
\begin{figure}[htbp!]
    \centering
    \begin{tikzpicture}[scale=0.75]
        \begin{scope}
            \clip (2,2) circle (1.5);
            \fill[shadow] (4,2) circle (1.5);
        \end{scope}
        \begin{scope}
            \clip (2,2) circle (1.5) (-0.5,0) rectangle (6.5,4);
            \fill[shadow] (4,2) circle (1.5);
        \end{scope}
        \begin{scope}
            \clip (4,2) circle (1.5) (-0.5,0) rectangle (6.5,4);
            \fill[shadow] (2,2) circle (1.5);
        \end{scope}

        \draw[thick] (-0.5,0) rectangle (6.5,4);
        \draw[thick] (2,2) circle (1.5);
        \draw[thick] (4,2) circle (1.5);

        \node[below left] at (6.5,4) {$U$};
        \node at (1.75,2) {$B$};
        \node at (4.25,2) {$A$};
    \end{tikzpicture}
    \caption{$A \cup B$ 的文氏图}
    \label{fig:并集文氏图}
\end{figure}

同交集一样,由此我们还可以定义一组集合的并集。
\begin{definition}[多个集合的并集]\label{def:多个并集}
    一组集合的交集是包含这组集合中所有成员集合的元素的集合。
\end{definition}

我们用符号
\begin{equation*}
    \bigcup_{i=1}^n A_i=A_1 \cup A_2 \cup \cdots \cup A_i
\end{equation*}
表示 $A_1$,$A_2$,$\cdots$,$A_n$ 的并集。
\begin{collections}
    \begin{example}
        已知 $A = \{ 1, 2, 3 \}$,$B = \{ 2, 3, 4\}$,求 $A \cup B$。
    \end{example}
    \begin{solution}
        $A \cap B = \{1, 2, 3, 4\}$
    \end{solution}

    \spare

    \begin{example}
        已知 $A = \{ x \mid 10 < x \leq 20 \}$,$B = \{ x \mid 0 < x \leq 10 \}$,求 $A \cup B$。
    \end{example}
    \begin{solution}
        $A \cup B = \{ x \mid 0 < x < 20\}$
    \end{solution}

    \spare

    \begin{example}
        已知 $A = \{ x \mid 10 < x \leq 20 \}$,$B = \{ x \mid 0 < x \leq 10 \}$,求 $A \cup B$。
    \end{example}
    \begin{solution}
        $A \cup B = \{ x \mid 0 < x < 20 \land x \neq 10 \}$
    \end{solution}
\end{collections}

\subsection{差集}
\begin{definition}[集合的差集]\label{def:差集}
    令 $A$ 与 $B$ 为集合,集合 $A$ 和 $B$ 的差集是一个集合,它包含属于 $A$ 但不属于 $B$ 的元素,记作 $A-B$。
    \begin{equation*}
        A - B = \{ x \mid x \in A \land x \notin B \}
    \end{equation*}
\end{definition}

集合 $A$ 与 $B$ 的差集可以用图 \ref{fig:差集文氏图} 中的阴影部分表示。
\begin{figure}[htbp!]
    \centering
    \begin{tikzpicture}[scale=0.75]
        \begin{scope}
            \clip (2,2) circle (1.5) (-0.5,0) rectangle (6.5,4);
            \fill[shadow] (4,2) circle (1.5);
        \end{scope}

        \draw[thick] (-0.5,0) rectangle (6.5,4);
        \draw[thick] (2,2) circle (1.5);
        \draw[thick] (4,2) circle (1.5);

        \node[below left] at (6.5,4) {$U$};
        \node at (1.75,2) {$B$};
        \node at (4.25,2) {$A$};
    \end{tikzpicture}
    \caption{$A - B$ 的文氏图}
    \label{fig:差集文氏图}
\end{figure}

\begin{collections}
    \begin{example}
        已知 $A = \{ 1, 2, 3 \}$,$B = \{2, 3, 4\}$,求 $A - B$。
    \end{example}
    \begin{solution}
        $A-B = \{1\}$
    \end{solution}

    \spare

    \begin{example}
        已知 $A = \{1, 2, 3\}$,$B = \{4, 5, 6\}$,求 $A - B$。
    \end{example}
    \begin{solution}
        $A - B = \{1, 2, 3\}$
    \end{solution}
\end{collections}

\subsection{补集}
\begin{definition}[集合的补集]
    令 $U$ 为全集,集合 $A$ 的补集就是 $U-A$,记作 $\bar{A}$
\end{definition}

集合 $A$ 的补集可以用图 \ref{fig:补集文氏图} 中的阴影部分表示。
\begin{figure}[htbp!]
    \centering
    \begin{tikzpicture}
        \begin{scope}
            \clip (0,0) rectangle (4,3) (2,1.5) circle (1);
            \fill[shadow] (0,0) rectangle (4,3);
        \end{scope}
        \draw[thick] (0,0) rectangle (4,3);
        \draw[thick] (2,1.5) circle (1);

        \node[below left] at (4,3) {$U$};
        \node at (2,1.5) {$A$};
    \end{tikzpicture}
    \caption{$\bar{A}$ 的文氏图}
    \label{fig:补集文氏图}
\end{figure}

\begin{collections}
    \begin{example}
        已知 $A = \{x \in \symbf{R} \mid 10 < x \leq 20\}$,求$\bar{A}$。
    \end{example}
    \begin{solution}
        $\bar{A} = \{x \in \symbf{R} \mid x \leq 10 \lor x > 20\}$
    \end{solution}

    \spare

    \begin{example}
        已知全集 $U$,$A = \varnothing$,求 $\bar{A}$。
    \end{example}
    \begin{solution}
        $\bar{A} = U$
    \end{solution}
\end{collections}

\subsection{对称差}
\begin{definition}[集合的对称差]\label{def:对称差}
    令 $A$ 与 $B$ 为集合,集合 $A$ 与 $B$ 的对称差是一个集合,由属于 $A$ 但不属于 $B$ 和属于 $B$ 但不属于 $A$ 的元素组成,记作 $A \oplus B$。
\end{definition}

集合 $A$ 与 $B$ 的对称差可以用图 \ref{fig:对称差文氏图} 中的阴影部分表示。
\begin{figure}[htbp!]
    \centering
    \begin{tikzpicture}[scale=0.75]
        \begin{scope}
            \clip (2,2) circle (1.5) (-0.5,0) rectangle (6.5,4);
            \fill[shadow] (4,2) circle (1.5);
        \end{scope}
        \begin{scope}
            \clip (4,2) circle (1.5) (-0.5,0) rectangle (6.5,4);
            \fill[shadow] (2,2) circle (1.5);
        \end{scope}

        \draw[thick] (-0.5,0) rectangle (6.5,4);
        \draw[thick] (2,2) circle (1.5);
        \draw[thick] (4,2) circle (1.5);

        \node[below left] at (6.5,4) {$U$};
        \node at (1.75,2) {$B$};
        \node at (4.25,2) {$A$};
    \end{tikzpicture}
    \caption{$A \oplus B$ 的文氏图}
    \label{fig:对称差文氏图}
\end{figure}

\begin{collections}
    \begin{example}
        已知 $A=\{1, 2, 3\}$,$B = \{2, 3, 4\}$,求 $A \oplus B$。
    \end{example}
    \begin{solution}
        $A \oplus B = \{1, 4\}$
    \end{solution}

    \spare

    \begin{example}
        已知 $A=\{1, 2, 3\}$,$B=\{1, 2, 3\}$,求 $A \oplus B$。
    \end{example}
    \begin{solution}
        $A \oplus B = \varnothing$
    \end{solution}
\end{collections}

\newpage

\section{图的概念}
\begin{introduction}
    \item 图
    \item 无向图、有向图与混合图
    \item 环、多重边
    \item 简单图、伪图与多重图
    \item 完全图、零图与平凡图
    \item 顶点与边的性质
    \item 子图与并图
\end{introduction}

\subsection{图}
图是由顶点与连接顶点的边构成的离散结构,图的定义如下
\begin{definition}[图]\label{def:图}
    图 $G = (V, E)$ 由顶点的非空集 $V$ 和边集 $E$ 构成,每条边有一个或两个顶点与之相连,这样的顶点叫做边的端点。
\end{definition}

图 \ref{fig:图} 展示了一个图。
\begin{figure}[H]
    \centering
    \begin{tikzpicture}
        \GraphInit[vstyle=normal]
        \SetGraphUnit{1.5}\SetVertexMath
        \Vertex{a}
        \EA(a){b}
        \EA(b){c}
        \EA(c){g}
        \NO(b){d}
        \NO(c){e}
        \Edge[style={->}](a)(b)
        \Edge[style={->}](b)(c)
        \Edge[style={->}](c)(g)
        \Edge[style={->}](a)(d)
        \Edge[style={->}](d)(e)
        \Edge[style={->}](e)(g)
        \Loop[dist=1.5cm, style={->}](a)
        \Loop[dist=1.5cm, dir=SO, style={->}](c)
    \end{tikzpicture}
    \caption{图}
    \label{fig:图}
\end{figure}

\subsection{无向图、有向图与混合图}
图可以根据边是否有方向这一特性来划分为无向图、有向图与混合图三类。
\begin{definition}[有向边与无向边]\label{def:无向边与有向边}
    无向边以基数为 2 的集合表示,$\{u, v\}$ 表示该无向边与顶点 $u$、$v$ 相关联。有向边以序偶表示,$(u, v)$ 表示该有向边与顶点 $u$、$v$ 相关联,且开始于 $u$,结束于 $v$。
\end{definition}

由此,如果图 $G$ 的边集 $E$ 中只有无向边,则该图称为无向图;如果图 $G$ 的边集 $E$ 中只有有向边,则该图称为有向图;如果图 $G$ 的边集 $E$ 中同时存在无向边与有向边,则该图称为混合图。
\begin{figure}[H]
    \centering
    \begin{tikzpicture}
        \GraphInit[vstyle=normal]
        \SetGraphUnit{1.5}\SetVertexMath
        \Vertex{a}
        \EA(a){b}
        \SO(b){c}
        \EA(b){d}
        \Edge[style={-}](a)(b)
        \Edge[style={-}](a)(c)
        \Edge[style={-}](b)(d)
        \Edge[style={-}](c)(d)
    \end{tikzpicture}
    \hspace{1em}
    \begin{tikzpicture}
        \GraphInit[vstyle=normal]
        \SetGraphUnit{1.5}\SetVertexMath
        \Vertex{a}
        \EA(a){b}
        \SO(b){c}
        \EA(b){d}
        \Edge[style={->}](a)(b)
        \Edge[style={->}](a)(c)
        \Edge[style={->}](b)(d)
        \Edge[style={->}](c)(d)
    \end{tikzpicture}
    \hspace{1em}
    \begin{tikzpicture}
        \GraphInit[vstyle=normal]
        \SetGraphUnit{1.5}\SetVertexMath
        \Vertex{a}
        \EA(a){b}
        \SO(b){c}
        \EA(b){d}
        \Edge[style={->}](a)(b)
        \Edge[style={-}](a)(c)
        \Edge[style={-}](b)(d)
        \Edge[style={->}](c)(d)
    \end{tikzpicture}
    \caption{无向图(左)、有向图(中)、混合图(右)}
\end{figure}

\subsection{环、多重边、简单图、伪图与多重图}
如果一条边所连接的顶点是同一个顶点,那么该边称为环。如果多条边连接同一对顶点,那么称其为多重边。
\vspace{-2em}
\begin{figure}[H]
    \centering
    \begin{tikzpicture}
        \GraphInit[vstyle=normal]
        \SetGraphUnit{1.5}\SetVertexMath
        \Vertex{a}
        \Loop[dist=1.5cm, style={->}](a)
    \end{tikzpicture}
    \hspace{1em}

    \begin{tikzpicture}
        \GraphInit[vstyle=normal]
        \SetGraphUnit{1.5}\SetVertexMath
        \Vertex{a}
        \EA(a){b}
        \Edge[style={-, bend left}](a)(b)
        \Edge[style={-}](a)(b)
    \end{tikzpicture}
    \caption{环(上)与多重边(下)}
    \label{fig:环与多重边}
\end{figure}

如果一个图不存在多重边或者环,那么该图被称为简单图;如果一个图存在多重边和环,那么该图被称为伪图;如果一个图存在多重边但不存在环,那么该图称为多重图。

\begin{table}[H]
    \centering
    \begin{tabular}{ccc}
        \toprule
        \makebox[2cm][c]{多重边} & \makebox[2cm][c]{环} & \makebox[2cm][c]{类型} \\
        \midrule
        不存在 & 不存在 & 简单图 \\
        存在 & 存在 & 伪图 \\
        存在 & 不存在 & 多重图 \\
        \bottomrule
    \end{tabular}
    \caption{简单图、伪图与多重图}
\end{table}
\vspace{-2em}
\begin{figure}[H]
    \centering
    \begin{tikzpicture}
        \GraphInit[vstyle=normal]
        \SetGraphUnit{1.5}\SetVertexMath
        \Vertex{a}
        \EA(a){b}
        \SO(b){c}
        \EA(b){d}
        \Edge[style={-}](a)(b)
        \Edge[style={-}](a)(c)
        \Edge[style={-}](b)(d)
        \Edge[style={-}](c)(d)
    \end{tikzpicture}
    \hspace{1em}
    \begin{tikzpicture}
        \GraphInit[vstyle=normal]
        \SetGraphUnit{1.5}\SetVertexMath
        \Vertex{a}
        \EA(a){b}
        \SO(b){c}
        \EA(b){d}
        \Edge[style={-}](a)(b)
        \Edge[style={-, bend left}](a)(b)
        \Edge[style={-, bend right}](a)(b)
        \Edge[style={-, bend right}](a)(c)
        \Edge[style={-}](a)(c)
        \Edge[style={-}](b)(d)
        \Edge[style={-}](c)(d)
        \Loop[dist=1.5cm, style={-}](a)
    \end{tikzpicture}
    \hspace{1em}
    \begin{tikzpicture}
        \GraphInit[vstyle=normal]
        \SetGraphUnit{1.5}\SetVertexMath
        \Vertex{a}
        \EA(a){b}
        \SO(b){c}
        \EA(b){d}
        \Edge[style={-}](a)(b)
        \Edge[style={-, bend left}](a)(b)
        \Edge[style={-, bend right}](a)(b)
        \Edge[style={-, bend right}](a)(c)
        \Edge[style={-}](a)(c)
        \Edge[style={-}](b)(d)
        \Edge[style={-}](c)(d)
    \end{tikzpicture}
    \caption{简单图(左)、伪图(中)、多重图(右)}
    \label{fig:简单图、伪图与多重图}
\end{figure}

\subsection{完全图、零图与平凡图}
完全图、零图与平凡图都是特殊的无向简单图。完全图指每对不同顶点之间都恰有一条边的简单图;零图指边集为空集的简单图;频繁图指只有一个顶点且边集为空集的简单图。
\begin{figure}[H]
    \centering
    \begin{tikzpicture}
        \GraphInit[vstyle=normal]
        \SetGraphUnit{1.5}\SetVertexMath
        \Vertex{a}
        \EA(a){b}
        \SO(a){c}
        \EA(c){d}
        \Edge[style={-}](a)(b)
        \Edge[style={-}](a)(c)
        \Edge[style={-}](a)(d)
        \Edge[style={-}](b)(c)
        \Edge[style={-}](b)(d)
        \Edge[style={-}](c)(d)
    \end{tikzpicture}
    \hspace{3em}
    \begin{tikzpicture}
        \GraphInit[vstyle=normal]
        \SetGraphUnit{1.5}\SetVertexMath
        \Vertex{a}
        \EA(a){b}
        \SO(a){c}
        \EA(c){d}
    \end{tikzpicture}
    \hspace{3em}
    \begin{tikzpicture}
        \GraphInit[vstyle=normal]
        \SetGraphUnit{1.5}\SetVertexMath
        \Vertex{a}
    \end{tikzpicture}
    \caption{完全图(左)、零图(中)与平凡图(右)}
    \label{完全图、零图与平凡图}
\end{figure}

\subsection{顶点与边的性质}
\begin{definition}[无向图顶点的邻接]\label{def:无向图邻接}
    若 $u$ 和 $v$ 是无向图 $G$ 中的一条边 $e$ 的端点,则称两个顶点 $u$ 和 $v$ 在 $G$ 里邻接,边 $e$ 关联(或连接)$u$ 和 $v$
\end{definition}
\begin{definition}[有向图顶点的邻接]\label{def:有向图邻接}
    若 $(u,v)$ 是有向图 $G$ 中的一条边,则称顶点 $u$ 邻接到 $v$,$v$ 从 $u$ 邻接,$u$ 称为边 $(u,v)$ 的起点,$v$ 称为边 $(u,v)$ 的终点。环的起点与终点相同。
\end{definition}
\begin{definition}[顶点的邻居]\label{def:邻居}
    图 $G=(V, E)$ 中,顶点 $v$ 的所有邻接顶点的集合称为顶点的邻居,记作 $N(v)$。
\end{definition}

\begin{collections}
    \begin{example}
        写出下图中各个顶点的邻居。
            \begin{center}
                \begin{tikzpicture}
                    \GraphInit[vstyle=normal]
                    \SetGraphUnit{1.5}\SetVertexMath
                    \Vertex{a}
                    \EA(a){b}
                    \SO(b){c}
                    \EA(b){d}
                    \Edge[style={-}](a)(b)
                    \Edge[style={-}](a)(c)
                    \Edge[style={-}](b)(d)
                    \Edge[style={-}](c)(d)
                    \Edge[style={-}](b)(c)
                    \Loop[dist=1cm, style={-}, dir=SO](c)
                \end{tikzpicture}
            \end{center}
    \end{example}
    \begin{solution}
        \begin{center}
            \begin{tabular}{c|c}
                \toprule
                \makebox[2cm][c]{顶点} & \makebox[2cm][c]{邻居} \\
                \midrule
                $a$ & $b, c$ \\
                $b$ & $a, c, d$ \\
                $c$ & $a, b, c, d$ \\
                $d$ & $b, c$ \\
                \bottomrule
            \end{tabular}
        \end{center}
    \end{solution}
\end{collections}

\begin{definition}[无向图顶点的度]
    无向图中顶点的度是与该顶点相关联的边的数目,如果该顶点存在环,则每个环为顶点的度贡献 $2$。顶点 $v$ 的度记作 $\symrm{deg}(v)$
\end{definition}
\begin{definition}[有向图顶点的出度与入度]
    在有向图中,顶点的出度是以该顶点为起点的边的数目,顶点的入度是以该顶点为终点的边的数目,顶点上的环对出度与入度的贡献均为 1。顶点 $v$ 的出度记为 $\symrm{deg}^+(v)$,顶点 $v$ 的入度记为 $\symrm{deg}^-(v)$。
\end{definition}

如果一个顶点的度为 $0$,那么称其为\textbf{孤立的};如果一个顶点的度为 $1$,那么称其为\textbf{悬挂的}。
\begin{collections}
    \begin{example}
        看下图,求$\symrm{deg}(c)$。
            \begin{center}
                \begin{tikzpicture}
                    \GraphInit[vstyle=normal]
                    \SetGraphUnit{1.5}\SetVertexMath
                    \Vertex{a}
                    \EA(a){b}
                    \SO(b){c}
                    \EA(b){d}
                    \Edge[style={-}](a)(b)
                    \Edge[style={-}](a)(c)
                    \Edge[style={-}](b)(d)
                    \Edge[style={-}](c)(d)
                    \Edge[style={-}](b)(c)
                    \Loop[dist=1cm, style={-}, dir=SO](c)
                \end{tikzpicture}
            \end{center}
    \end{example}
    \begin{solution}
        $\symrm{deg}(c)=5$
    \end{solution}
    \spare
    \begin{example}
        看下图,求 $\symrm{deg}^+(c)$ 和 $\symrm{deg}^-(c)$。
            \begin{center}
                \begin{tikzpicture}
                    \GraphInit[vstyle=normal]
                    \SetGraphUnit{1.5}\SetVertexMath
                    \Vertex{a}
                    \EA(a){b}
                    \SO(b){c}
                    \EA(b){d}
                    \Edge[style={->}](a)(b)
                    \Edge[style={->}](a)(c)
                    \Edge[style={->}](b)(d)
                    \Edge[style={->}](c)(d)
                    \Edge[style={->}](b)(c)
                    \Loop[dist=1cm, style={->}, dir=SO](c)
                \end{tikzpicture}
            \end{center}
    \end{example}
    \begin{solution}
        $\symrm{deg}^+(c)=2$,$\symrm{deg}^-(c)=3$
    \end{solution}
\end{collections}

\begin{theorem}[无向图的握手定理]\label{thm:无向图的握手定理}
    设 $G=(V,E)$ 是有 $m$ 条边的无向图,则
    \begin{equation*}
        2m = \sum_{v \in V} \symrm{deg}(v)
    \end{equation*}
\end{theorem}
\begin{theorem}[有向图的握手定理]\label{thm:有向图的握手定理}
    设 $G=(V,E)$ 是有 $m$ 条边的有向图,则
    \begin{equation*}
        m = \sum_{v \in V} \symrm{deg}^+(v) = \sum_{v \in V} \symrm{deg}^-(v)
    \end{equation*}
\end{theorem}

握手定理在图中存在环或多重边时依然成立。

由定理 \ref{thm:无向图的握手定理} 可以推得定理 \ref{thm:握手定理推论}。
\begin{theorem}\label{thm:握手定理推论}
    无向图有偶数个度为奇数的顶点。
\end{theorem}

\begin{collections}
    \begin{example}
        如果一个无向图中有 6 个顶点,每个顶点的度均为 10,则该图中有多少条边?
    \end{example}
    \begin{solution}
        设该图为 $G=(V, E)$,有 $m$ 条边,由题意可知

        $$\sum_{v \in V}\symrm{deg}(v) = 6 \times 10 = 60$$

        由握手定理可知

        $$\sum_{v \in V}\symrm{deg}(v) = 2m$$

        则

        $$2m = 60$$

        解得 $m=30$

        综上所述,该图有 $30$ 条边。
    \end{solution}
\end{collections}

\subsection{子图与并图}
\begin{definition}[子图]\label{def:子图}
    设图 $G=(V, E)$ 与图 $H=(W, F)$ 存在 $W \subseteq V$ 与 $E \subseteq F$,则称图 $H$ 是图 $G$ 的子图,记作 $H \subseteq G$。
\end{definition}
\begin{definition}[真子图]\label{def:真子图}
    设图 $G=(V, E)$ 与图 $H=(W, F)$ 存在 $W \subseteq V$ 与 $E \subseteq F$,且$H \neq G$,则称图 $H$ 是图 $G$ 的真子图,记作 $H \subsetneqq G$。
\end{definition}
\begin{definition}[生成子图]\label{def:生成子图}
    设图 $G=(V, E)$ 与图 $H=(W, F)$ 存在 $V = W$ 与 $E \subseteq F$,则称图 $H$ 是图 $G$ 的生成子图。
\end{definition}

显然,任意图 $G$ 都是自身的子图和生成子图。
\begin{figure}[htbp!]
    \centering
    \begin{tikzpicture}
        \GraphInit[vstyle=normal]
        \SetGraphUnit{1.5}\SetVertexMath
        \Vertex{a}
        \EA(a){b}
        \SO(a){c}
        \EA(c){d}
        \Edge[style={-}](a)(b)
        \Edge[style={-}](a)(c)
        \Edge[style={-}](a)(d)
        \Edge[style={-}](b)(c)
        \Edge[style={-}](b)(d)
        \Edge[style={-}](c)(d)
    \end{tikzpicture}
    \hspace{3em}
    \begin{tikzpicture}
        \GraphInit[vstyle=normal]
        \SetGraphUnit{1.5}\SetVertexMath
        \Vertex{a}
        \EA(a){b}
        \SO(a){c}
        \Edge[style={-}](a)(b)
        \Edge[style={-}](a)(c)
        \Edge[style={-}](b)(c)
    \end{tikzpicture}
    \hspace{3em}
    \begin{tikzpicture}
        \GraphInit[vstyle=normal]
        \SetGraphUnit{1.5}\SetVertexMath
        \Vertex{a}
        \EA(a){b}
        \SO(a){c}
        \Edge[style={-}](a)(b)
        \Edge[style={-}](a)(c)
    \end{tikzpicture}
    \hspace{3em}
    \begin{tikzpicture}
        \GraphInit[vstyle=normal]
        \SetGraphUnit{1.5}\SetVertexMath
        \Vertex{a}
        \EA(a){b}
        \SO(a){c}
        \EA(c){d}
        \Edge[style={-}](a)(b)
        \Edge[style={-}](a)(c)
        \Edge[style={-}](b)(d)
        \Edge[style={-}](c)(d)
    \end{tikzpicture}
    \caption{图 $G$(左一) 与他的子图(左二)、真子图(右二)、生成子图(右一)}
    \label{fig:子图、真子图与生成子图}
\end{figure}

\begin{definition}[并图]\label{def:并图}
    设 $G_1=(V_1,E_1)$ 和 $G_2=(V_2,E_2)$ 是简单图,$G_1$ 和 $G_2$ 的并图是带有顶点集 $V_1 \cup V_2$ 和边集 $E_1 \cup E_2$ 的简单图,记作 $G_1 \cup G_2$。
\end{definition}

\begin{collections}
    \begin{example}
        求下图中 $G_1$ 和 $G_2$ 的并图 $G_1 \cup G_2$。
        \begin{center}
            \begin{tikzpicture}
                \GraphInit[vstyle=normal]
                \SetGraphUnit{1.5}\SetVertexMath
                \Vertex{a}
                \EA(a){b}
                \EA(b){c}
                \SO(a){d}
                \EA(d){e}
                \Edge[style={-}](a)(b)
                \Edge[style={-}](a)(d)
                \Edge[style={-}](b)(c)
                \Edge[style={-}](b)(e)
                \Edge[style={-}](c)(e)
                \Edge[style={-}](d)(e)
                \node[below of=b, yshift=-1.5cm]{$G_1$};
            \end{tikzpicture}
            \hspace{5em}
            \begin{tikzpicture}
                \GraphInit[vstyle=normal]
                \SetGraphUnit{1.5}\SetVertexMath
                \Vertex{a}
                \EA(a){b}
                \EA(b){c}
                \SO(a){d}
                \SO(c){f}
                \Edge[style={-}](a)(b)
                \Edge[style={-}](b)(c)
                \Edge[style={-}](b)(d)
                \Edge[style={-}](b)(f)
                \Edge[style={-}](c)(f)
                \node[below of=b, yshift=-1.5cm]{$G_2$};
            \end{tikzpicture}
        \end{center}
    \end{example}
    \begin{solution}
        \begin{center}
            \begin{tikzpicture}
                \GraphInit[vstyle=normal]
                \SetGraphUnit{1.5}\SetVertexMath
                \Vertex{a}
                \EA(a){b}
                \EA(b){c}
                \SO(b){e}
                \SO(a){d}
                \SO(c){f}
                \Edge[style={-}](a)(b)
                \Edge[style={-}](a)(d)
                \Edge[style={-}](b)(c)
                \Edge[style={-}](b)(d)
                \Edge[style={-}](b)(e)
                \Edge[style={-}](b)(f)
                \Edge[style={-}](c)(e)
                \Edge[style={-}](c)(f)
                \node[below of=b, yshift=-1.5cm]{$G_1 \cup G_2$};
            \end{tikzpicture}
        \end{center}
    \end{solution}
\end{collections}

\newpage
\section{图的表示}
\begin{introduction}
    \item 邻接表
    \item 邻接矩阵
    \item 关联矩阵
\end{introduction}
\subsection{邻接表}
邻接表可以用来表示不存在多重边的图。
\begin{collections}
    \begin{example}
        用邻接表表示以下无向图。
        \vspace{-2em}
        \begin{center}
            \begin{tikzpicture}
                \GraphInit[vstyle=normal]
                \SetGraphUnit{1.5}\SetVertexMath
                \Vertex{a}
                \EA(a){b}
                \SO(a){c}
                \EA(c){d}
                \EA(d){e}
                \Edge[style={-}](a)(b)
                \Edge[style={-}](a)(c)
                \Edge[style={-}](a)(d)
                \Edge[style={-}](b)(d)
                \Edge[style={-}](b)(e)
                \Edge[style={-}](c)(d)
                \Edge[style={-}](d)(e)
                \Loop[style={-},dist=1.5cm](a)
            \end{tikzpicture}
        \end{center}
    \end{example}
    \begin{solution}
        \begin{center}
            \begin{tabular}{c|c}
                \toprule
                \makebox[2cm][c]{顶点} & \makebox[2cm][c]{邻居} \\
                \midrule
                $a$ & $a, b, c, d$ \\
                $b$ & $a, d, e$ \\
                $c$ & $a, d$ \\
                $d$ & $a, b, c, e$ \\
                $e$ & $b, d$\\
                \bottomrule
            \end{tabular}
        \end{center}
    \end{solution}

    \spare

    \begin{example}
        用邻接表表示以下有向图。
        \vspace{-2em}
        \begin{center}
            \begin{tikzpicture}
                \GraphInit[vstyle=normal]
                \SetGraphUnit{1.5}\SetVertexMath
                \Vertex{a}
                \EA(a){b}
                \SO(a){c}
                \EA(c){d}
                \EA(d){e}
                \Edge[style={->, bend left}](a)(b)
                \Edge[style={->, bend left}](b)(a)
                \Edge[style={->}](a)(c)
                \Edge[style={->}](a)(d)
                \Edge[style={->}](b)(d)
                \Edge[style={->}](b)(e)
                \Edge[style={->}](c)(d)
                \Edge[style={->}](d)(e)
                \Loop[style={->},dist=1.5cm](a)
                \Loop[style={->},dist=1.5cm, dir=EA](e)
            \end{tikzpicture}
            \vspace{-2em}
        \end{center}
    \end{example}
    \begin{solution}
        \begin{center}
            \begin{tabular}{c|c}
                \toprule
                \makebox[2cm][c]{起点} & \makebox[2cm][c]{终点} \\
                \midrule
                $a$ & $a, b, c, d$ \\
                $b$ & $a, d, e$ \\
                $c$ & $d$ \\
                $d$ & $e$ \\
                $e$ & $e$ \\
                \bottomrule
            \end{tabular}
        \end{center}
    \end{solution}
\end{collections}

\subsection{邻接矩阵}
当图中边的数量较多时,可以使用邻接矩阵来表示图。

无向图 $G=(V, E)$ 的邻接矩阵是一个大小为 $|V| \times |V|$ 的矩阵,矩阵中的元素 $a_{ij}$ 表示顶点 $v_i$ 与 $v_j$ 之间边的数量。无向图的邻接矩阵是关于其主对角线对称的。

有向图 $G=(V, E)$ 的邻接矩阵也是一个大小为 $|V| \times |V|$ 的矩阵,不过矩阵中的元素 $a_{ij}$ 表示以顶点 $v_i$ 为起点,$v_j$ 为终点的边的数量。有向图的邻接矩阵不一定是对称的。
\begin{collections}
    \begin{example}
        用邻接矩阵表示以下无向图。
        \vspace{-2em}
        \begin{center}
            \begin{tikzpicture}
                \GraphInit[vstyle=normal]
                \SetGraphUnit{1.5}\SetVertexMath
                \Vertex{a}
                \EA(a){b}
                \SO(a){c}
                \EA(c){d}
                \EA(d){e}
                \Edge[style={-}](a)(b)
                \Edge[style={-}](a)(c)
                \Edge[style={-}](a)(d)
                \Edge[style={-}](b)(d)
                \Edge[style={-}](b)(e)
                \Edge[style={-}](c)(d)
                \Edge[style={-}](d)(e)
                \Loop[style={-},dist=1.5cm](a)
            \end{tikzpicture}
        \end{center}
    \end{example}
    \begin{solution}
        $$
        \begin{bmatrix}
            1 & 1 & 1 & 1 & 0 \\
            1 & 0 & 0 & 1 & 1 \\
            1 & 0 & 0 & 1 & 0 \\
            1 & 1 & 1 & 0 & 1 \\
            0 & 1 & 0 & 1 & 0 \\
        \end{bmatrix}
        $$
    \end{solution}

    \spare

    \begin{example}
        用邻接矩阵表示以下有向图。
        \vspace{-2em}
        \begin{center}
            \begin{tikzpicture}
                \GraphInit[vstyle=normal]
                \SetGraphUnit{1.5}\SetVertexMath
                \Vertex{a}
                \EA(a){b}
                \SO(a){c}
                \EA(c){d}
                \EA(d){e}
                \Edge[style={->}](a)(b)
                \Edge[style={->, bend left}](a)(b)
                \Edge[style={->, bend left}](b)(a)
                \Edge[style={->}](a)(c)
                \Edge[style={->}](a)(d)
                \Edge[style={->}](b)(d)
                \Edge[style={->}](b)(e)
                \Edge[style={->}](c)(d)
                \Edge[style={->}](d)(e)
                \Loop[style={->},dist=1.5cm](a)
                \Loop[style={->},dist=1.5cm, dir=EA](e)
            \end{tikzpicture}
            \vspace{-2em}
        \end{center}
    \end{example}
    \begin{solution}
        $$
        \begin{bmatrix}
            1 & 2 & 1 & 1 & 0 \\
            1 & 0 & 0 & 1 & 1 \\
            0 & 0 & 0 & 1 & 0 \\
            0 & 0 & 0 & 0 & 1 \\
            0 & 0 & 0 & 0 & 1 \\
        \end{bmatrix}
        $$
    \end{solution}
\end{collections}

可以看出,对于无向图的邻接矩阵,存在
\begin{equation*}
    \sum_{j=1}^{|V|}(a_{ij} + a_{ji})=\symrm{deg}(v_i)
\end{equation*}
对于有向图的邻接矩阵,存在
\begin{equation*}
    \begin{aligned}
        \sum_{j=1}^{|V|}a_{ij}=\symrm{deg}^+(v_i) \\
        \sum_{j=1}^{|V|}a_{ji}=\symrm{deg}^-(v_i) \\
    \end{aligned}
\end{equation*}

\subsection{关联矩阵}
关联矩阵于邻接矩阵不同,关联矩阵只能用于表示无向图。设无向图 $G=(V,E)$,其关联矩阵是一个大小为 $|V|\times|E|$ 的 $01$ 矩阵,矩阵中的元素 $a_{ij}$ 有以下特点。
\begin{equation*}
    a_{ij}=
    \begin{cases}
        1 & \text{$v_i$ 与 $e_j$ 相关联} \\
        0 & \text{$v_i$ 与 $e_j$ 无关} \\
    \end{cases}
\end{equation*}

\begin{collections}
    \begin{example}
        用关联矩阵表示以下无向图。
        \vspace{-2em}
        \begin{center}
            \begin{tikzpicture}
                \GraphInit[vstyle=normal]
                \SetGraphUnit{1.5}\SetVertexMath
                \Vertex{a}
                \EA(a){b}
                \SO(a){c}
                \EA(c){d}
                \EA(d){e}
                \Edge[style={-}, label={$v_1$}](a)(b)
                \Edge[style={-}, label={$v_2$}](a)(c)
                \Edge[style={-}, label={$v_3$}](a)(d)
                \Edge[style={-}, label={$v_4$}](b)(d)
                \Edge[style={-}, label={$v_5$}](b)(e)
                \Edge[style={-}, label={$v_6$}](c)(d)
                \Edge[style={-}, label={$v_7$}](d)(e)
                \Loop[style={-}, labelstyle={fill=white}, label={$v_8$}, dist=1.5cm](a)
            \end{tikzpicture}
        \end{center}
        \begin{solution}
            $$
            \begin{bmatrix}
                1 & 1 & 1 & 0 & 0 & 0 & 0 & 1 \\
                1 & 0 & 0 & 1 & 1 & 0 & 0 & 0 \\
                0 & 1 & 0 & 0 & 0 & 1 & 1 & 0 \\
                0 & 0 & 1 & 1 & 0 & 1 & 0 & 0 \\
                0 & 0 & 0 & 0 & 1 & 0 & 1 & 0 \\
            \end{bmatrix}
            $$
        \end{solution}
    \end{example}
\end{collections}

\newpage

\section{图的连通性}
\begin{introduction}
    \item 通路与回路
    \item 无向图的连通性
    \item 有向图的连通性
\end{introduction}

\subsection{通路与回路}

\begin{definition}[无向图的通路与回路]\label{def:无向图的通路与回路}
    设 $n$ 是正整数且 $G=(V,E)$ 是无向图,$u,v \in V$。在 $G$ 中从 $u$ 到 $v$ 的长度为 $n$ 的通路指 $G$ 中边的序列 $(e_1,e_2,\cdots,e_n)$,其中边 $e_1$ 与顶点 $u$ 相关联,边 $e_n$ 与顶点 $v$ 相关联,根据这个边的序列,可以列出与每条边相关联的顶点序列 $(u,\cdots,v)$,当 $G$ 是简单无向图时,可以使用该顶点顶点序列来表示这条通路。如果一条通路在相同的顶点上开始和结束且该通路长度大于 $0$,则称这条通路是一条回路。若通路或回路中不包含相同的边,则称他是简单的。
\end{definition}

\begin{collections}
    \begin{example}
        请指出下图中任意一条回路和从顶点 $a$ 到 $e$ 的任意一条通路。
        \begin{center}
            \begin{tikzpicture}
                \GraphInit[vstyle=normal]
                \SetGraphUnit{1.5}\SetVertexMath
                \Vertex{a}
                \EA(a){b}
                \SO(a){c}
                \EA(c){d}
                \EA(d){e}
                \Edge[style={-}](a)(b)
                \Edge[style={-}](a)(c)
                \Edge[style={-}](a)(d)
                \Edge[style={-}](b)(d)
                \Edge[style={-}](b)(e)
                \Edge[style={-}](c)(d)
                \Edge[style={-}](d)(e)
            \end{tikzpicture}
        \end{center}
    \end{example}
    \begin{solution}
        上图中存在回路 $(a, b, d, a)$ 及从顶点 $a$ 到 $e$ 的通路 $(a, d, e)$。
        \begin{center}
            \begin{tikzpicture}
                \SetGraphUnit{1.5}\SetVertexMath
                \Vertex{a}
                \EA(a){b}
                \SO(a){c}
                \EA(c){d}
                \EA(d){e}
                \Edge[style={-}, color=third](a)(b)
                \Edge[style={-}](a)(c)
                \Edge[style={-}, color=third](a)(d)
                \Edge[style={-}, color=third](b)(d)
                \Edge[style={-}](b)(e)
                \Edge[style={-}](c)(d)
                \Edge[style={-}](d)(e)
            \end{tikzpicture}
            \hspace{3em}
            \begin{tikzpicture}
                \SetGraphUnit{1.5}\SetVertexMath
                \Vertex{a}
                \EA(a){b}
                \SO(a){c}
                \EA(c){d}
                \EA(d){e}
                \Edge[style={-}](a)(b)
                \Edge[style={-}](a)(c)
                \Edge[style={-}, color=third](a)(d)
                \Edge[style={-}](b)(d)
                \Edge[style={-}](b)(e)
                \Edge[style={-}](c)(d)
                \Edge[style={-}, color=third](d)(e)
            \end{tikzpicture}
        \end{center}
    \end{solution}
\end{collections}

由定义 \ref{def:无向图的通路与回路} 我们可以类比作出有向图中通路与回路的定义。

\begin{definition}[有向图的通路与回路]\label{def:有向图的通路与回路}
    设 $n$ 是正整数且 $G=(V,E)$ 是有向图,$x_1,x_2,x_{n-1},x_n \in V$。在 $G$ 中从 $x_1$ 到 $x_n$ 的长度为 $n$ 的通路指 $G$ 中边的序列 $(e_1,e_2,\cdots,e_n)$,其中边 $e_1 = (x_1,x_2)$,边 $e_n = (x_{n-1},x_n)$,根据这个边的序列,可以列出与每条边相关联的顶点序列 $(x_1,x_2,\cdots,x_n)$,当 $G$ 是简单有向图时,可以使用该顶点顶点序列来表示这条通路。如果一条通路在相同的顶点上开始和结束且该通路长度大于 $0$,则称这条通路是一条回路。若通路或回路中不包含相同的边,则称他是简单的。
\end{definition}



\subsection{无向图的连通性}
\begin{definition}[无向图的连通性]\label{def:无向图的连通性}
    若无向图中每一对不同的顶点之间都有通路,则称该图为连通的,否则称为不连通的。
\end{definition}

\end{document}