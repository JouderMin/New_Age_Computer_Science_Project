\documentclass[lang=cn, chinesefont=founder, math=mtpro2, color=cyan, citestyle=gb7714-2015, bibstyle=gb7714-2015]{elegantbook}

\ExecuteBibliographyOptions{sorting=gb7714-2015}

\setlength{\parskip}{1ex plus 0.5ex minus 0.2ex}

\title{新世代计算机科学计划·基础篇}
\author{JouderMin}
\institute{「新世代计算机科学计划」制作委员会}
\date{\zhtoday}
\cover{img/cover.png}
\logo{img/方形logo.png}

\begin{document}
\maketitle
\frontmatter

\tableofcontents

\mainmatter

\chapter{计算机基础概述}
\section{计算机的组成}

计算机是一种用于进行高速计算的现代机器,由\emph{硬件系统}与\emph{软件系统}构成。目前现代计算机基本上均采用冯·诺依曼提出的存储程序结构(该结构又称冯·诺伊曼结构)。

存储程序结构有以下特点:
\begin{enumerate}
    \item 以运算单元为中心;
    \item 采用存储程序原理;
    \item 存储器是按地址访问、线性编址的空间;
    \item 控制流由指令流产生;
    \item 指令由操作码和地址码组成;
    \item 数据以二进制编码。
\end{enumerate}

\end{document}