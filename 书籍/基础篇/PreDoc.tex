\usepackage{unicode-math}
\setmathfont{STIXTwoMath-Regular.otf}

% \newcommand{\symbf}[1]{\mathbf{#1}}
% \newcommand{\symrm}[1]{\mathrm{#1}}
% \newcommand{\symcal}[1]{\mathcal{#1}}

\setmainfont{STIXTwoText}[
    UprightFont = *-Regular ,
    BoldFont = *-Bold ,
    ItalicFont = *-Italic ,
    BoldItalicFont = *-BoldItalic ,
    Extension = .otf ,
    Scale = 1.0]

\setCJKmainfont{SourceHanSerifSC}[
   Path = ./fonts/ ,
   UprightFont = *-Regular.otf ,
   BoldFont = *-Bold.otf ,
   ItalicFont = LXGWWenKai-Regular.ttf ,
   BoldItalicFont = LXGWWenKai-Bold.ttf ,
   Scale = 1.0]

\setCJKsansfont{SourceHanSansSC}[
   Path = ./fonts/ ,
   UprightFont = *-Regular.otf ,
   BoldFont = *-Bold.otf ,
   ItalicFont = LXGWWenKai-Regular.ttf ,
   BoldItalicFont = LXGWWenKai-Bold.ttf ,
   Scale = 1.0]

\setCJKmonofont{LXGWWenKaiMono}[
   Path = ./fonts/ ,
   UprightFont = *-Regular ,
   BoldFont = *-Bold ,
   Extension = .ttf ,
   Scale = 1.0]

\setCJKfamilyfont{zhsong}{SourceHanSerifSC}[
    Path = ./fonts/ ,
    UprightFont = *-Regular.otf ,
    BoldFont = *-Bold.otf ,
    ItalicFont = LXGWWenKai-Regular.ttf ,
    BoldItalicFont = LXGWWenKai-Bold.ttf ,
    Scale = 1.0]
\setCJKfamilyfont{zhhei}{SourceHanSansSC}[
    Path = ./fonts/ ,
    UprightFont = *-Regular.otf ,
    BoldFont = *-Bold.otf ,
    ItalicFont = LXGWWenKai-Regular.ttf ,
    BoldItalicFont = LXGWWenKai-Bold.ttf ,
    Scale = 1.0]
\setCJKfamilyfont{zhkai}{LXGWWenKai}[
    Path = ./fonts/ ,
    UprightFont = *-Regular ,
    BoldFont = *-Bold ,
    Extension = .ttf ,
    Scale = 1.0]
\setCJKfamilyfont{zhfs}{LXGWWenKaiMono}[
    Path = ./fonts/ ,
    UprightFont = *-Regular ,
    BoldFont = *-Bold ,
    Extension = .ttf ,
    Scale = 1.0]
\renewcommand*{\songti}{\CJKfamily{zhsong}}
\renewcommand*{\heiti}{\CJKfamily{zhhei}}
\renewcommand*{\kaishu}{\CJKfamily{zhkai}}
\renewcommand*{\fangsong}{\CJKfamily{zhfs}}

\usepackage{minted}
\usepackage[color=black]{siunitx}
\usepackage[ISO]{diffcoeff}
\usepackage{tikz}
\usepackage{tkz-euclide}
\usepackage{tkz-graph}
\usepackage{dashrule}
\usepackage{float}
\usepackage{fontawesome}
\usepackage{multirow}
\usepackage{longtable}

\ctexset{
    punct=kaiming
}

\newcommand{\qed}{\hfill$\QED$}

\PassOptionsToPackage{dvipsnames,svgnames,x11names}{xcolor}

\definecolor{shadow}{RGB}{210,241,241}
\definecolor{information}{RGB}{220,234,247}
\definecolor{infotitle}{RGB}{34,64,113}

\titleformat{\part}[display]{\vspace*{\fill}\bfseries}{
  \hspace*{\fill}\zihao{1}\enspace\bfseries{\color{structurecolor} \thepart}}{0pt}{
  \hspace*{\fill}\zihao{-0}\bfseries\color{black}}[\vspace{-1em}\color{structurecolor}\rule{\textwidth}{2pt}\vspace*{\fill}\newpage]
% \titleformat{\section}[hang]{\vspace*{0.5em}\bfseries}{
%   \filcenter\LARGE\bfseries{\color{structurecolor}\thesection}\enspace}{0pt}{%
%   \color{structurecolor}\LARGE\bfseries\filcenter}[\vspace{-1.3em}\hspace*{\fill}\color{structurecolor}\rule{0.2\textwidth}{1pt}\hspace*{\fill}\vspace*{1.3em}]
% \titleformat{\subsection}[hang]{\bfseries}{
%   \Large\bfseries\color{structurecolor}\thesubsection\enspace}{1pt}{%
%   \color{structurecolor}\large\bfseries\filright}
% \titleformat{\subsubsection}[hang]{\bfseries}{
%   \large\bfseries\color{structurecolor}\thesubsubsection\enspace}{1pt}{%
%   \color{structurecolor}\large\bfseries\filright}

\tcbuselibrary{minted}
\newcommand{\spare}{\vspace{-1em}\begin{center}\color{structurecolor}\hdashrule[0.5ex]{\textwidth}{1pt}{1pt}\end{center}\vspace{-1em}}
\usetikzlibrary{shapes,backgrounds}
\ExecuteBibliographyOptions{sorting=gb7714-2015}
\setlength{\parskip}{1ex}
\newtcolorbox{collections}{
      boxrule=0.5pt,
      enhanced,
      breakable,
      top=8pt,
      before skip=8pt,
      colframe=structurecolor,
      colback=structurecolor!5,
      colbacktitle=structurecolor
}
\newtcolorbox{info}[1]{
      boxrule=0.5pt,
      enhanced,
      breakable,
      top=8pt,
      before skip=8pt,
      title = \vspace*{7pt}\color{infotitle}\faInfoCircle\enspace\textbf{#1},
      colframe=information,
      colback=information,
      colbacktitle=information
}
\newtcblisting{code*}[1]{
    listing engine=minted,
    boxrule=0.1mm,
    colback=third!5,
    colframe=third,
    listing only,
    left=5mm,
    enhanced,
    breakable,
    overlay={\begin{tcbclipinterior}\fill[third] (frame.south west)
    rectangle ([xshift=5mm]frame.north west);\end{tcbclipinterior}},
    minted language=#1,
    minted style=vs,
    minted options={breaklines, mathescape, autogobble, linenos, numbersep=2mm}
}
\newtcblisting[auto counter]{code}[2]{
    listing engine=minted,
    boxrule=0.1mm,
    colback=third!5,
    colframe=third,
    listing only,
    left=5mm,
    enhanced,
    breakable,
    title=Program \Alph{\tcbcounter}: #2,
    overlay={\begin{tcbclipinterior}\fill[third] (frame.south west)
    rectangle ([xshift=5mm]frame.north west);\end{tcbclipinterior}},
    minted language=#1,
    minted style=vs,
    minted options={breaklines, mathescape, autogobble, linenos, numbersep=2mm}
}
\renewcommand{\theFancyVerbLine}{\textcolor{white}{\footnotesize{\arabic{FancyVerbLine}}}}

\renewcommand{\bar}[1]{\overline{#1}}
\arrayrulecolor{second}

\usetikzlibrary{graphs}