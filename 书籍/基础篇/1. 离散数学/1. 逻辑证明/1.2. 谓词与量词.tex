\section{谓词与量词}
\begin{introduction}
    \item 谓词
    \item 量词
    \item 嵌套量词
\end{introduction}

\subsection{谓词}
在命题演算之中,我们可以用形如 $R(x)$ 的形式来表示变量 $x$ 的性质,当然,其真值会根据 $x$ 的取值而变化。$R(x)$ 便叫做\emph{谓词}。
\begin{collections}
    \begin{example}
        令 $R(x)$ 表示语句“$x > 0$”,求 $R(-1)$ 与 $R(2)$ 的真值。
    \end{example}
    \begin{solution}
        当 $x = -1$ 时,$-1 \ngtr 0$,因此 $R(-1)$ 的真值为 F;当 $x = 2$ 时,$2 > 0$,因此 $R(2)$ 的真值为 T。
    \end{solution}
\end{collections}

事实上,谓词不单单能表示一个变量的性质,它可以表示多个变量的性质,例如我们令 $Q(x,y)$ 表示语句“$y = x + 1$”,显然这个谓词同时表示了变量 $x$ 与 $y$ 的性质,我们称这样的谓词为\emph{二元谓词}。接下来我们将谓词推广到可以表示 $n$ 个变量的性质,那么涉及 $n$ 个变量的谓词我们一般表示为
\begin{equation*}
    P(x_1,x_2,\cdots,x_n)
\end{equation*}
我们称这样的谓词为 \emph{$n$ 元谓词}。

\subsection{量词}
在命题演算中最常见的量化一般有 2 种,全称量化与存在量化,他们表示一个谓词在考虑范围内的真值情况,这个考虑范围我们称为论域,如果没有指定论域,那么全称量化和存在量化均无意义。

\begin{definition}[全称量化]\label{def:全称量化}
    $P(x)$ 的全称量化为语句“论域内的所有 $x$ 满足 $P(x)$”,记作 $\forall x P(x)$,其中 $\forall$ 是全称量词。
\end{definition}

\begin{definition}[存在量化]\label{def:存在量化}
    $P(x)$ 的存在量化为语句“论域内存在一个 $x$ 满足 $P(x)$”,记作 $\exists x P(x)$,其中 $\exists$ 是存在量词。
\end{definition}

\begin{collections}
    \begin{example}
        令 $P(x)$ 表示语句“$x + 1 > x$”,论域为全体实数集,求 $\forall x P(x)$ 的真值。
    \end{example}
    \begin{solution}
        显然 $P(x)$ 对所有实数均为真,所以 $\forall x P(x)$ 的值为真。
    \end{solution}

    \spare

    \begin{example}
        令 $P(x)$ 表示语句 $x = 2$,论域为 $x < 10$,求 $\exists x P(x)$ 的真值。
    \end{example}
    \begin{solution}
        在论域 $x < 10$ 中存在 $x = 2$ 使得 $P(x)$ 成立,所以 $\exists x P(x)$ 的值为真。
    \end{solution}
\end{collections}

除了以上的两种量化,还存在一个较少使用的量化,唯一性量化。
\begin{definition}
    $P(x)$ 的唯一性量化为语句“论域内存在且只存在一个 $x$ 满足 $P(x)$”,记作 $\exists ! x P(x)$,其中 $\exists !$ 是唯一性量词。
\end{definition}

事实上唯一性量化的使用是可以被避免的。
\begin{theorem}
    $\exists ! x P(x) \iff \exists x P(x) \land \forall x \forall y(P(x) \land P(y) \to (x = y))$
\end{theorem}