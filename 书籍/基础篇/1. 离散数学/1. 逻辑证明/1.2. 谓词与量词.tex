\section{谓词与量词}
\begin{introduction}
    \item 谓词
    \item 量词
\end{introduction}

\subsection{谓词}
在命题演算之中,我们可以用形如 $R(x)$ 的形式来表示变量 $x$ 的性质,当然,其真值会根据 $x$ 的取值而变化。$R(x)$ 便叫做\emph{谓词}。
\begin{collections}
    \begin{example}
        令 $R(x)$ 表示语句“$x > 0$”,求 $R(-1)$ 与 $R(2)$ 的真值。
    \end{example}
    \begin{solution}
        当 $x = -1$ 时,$-1 \ngtr 0$,因此 $R(-1)$ 的真值为 F;当 $x = 2$ 时,$2 > 0$,因此 $R(2)$ 的真值为 T。
    \end{solution}
\end{collections}

事实上,谓词不单单能表示一个变量的性质,它可以表示多个变量的性质,例如我们令 $Q(x,y)$ 表示语句“$y = x + 1$”,显然这个谓词同时表示了变量 $x$ 与 $y$ 的性质,我们称这样的谓词为\emph{二元谓词}。接下来我们将谓词推广到可以表示 $n$ 个变量的性质,那么涉及 $n$ 个变量的谓词我们一般表示为
\begin{equation*}
    P(x_1,x_2,\cdots,x_n)
\end{equation*}
我们称这样的谓词为 \emph{$n$ 元谓词}。

\subsection{量词}
在命题演算中最常见的量化一般有 2 种,全称量化与存在量化,他们表示一个谓词在考虑范围内的真值情况,这个考虑范围我们称为论域,如果没有指定论域,那么全称量化和存在量化均无意义。量词拥有比所有逻辑运算符都高的运算优先级。

\begin{definition}[全称量化]\label{def:全称量化}
    $P(x)$ 的全称量化为语句“论域内的所有 $x$ 满足 $P(x)$”,记作 $\forall x P(x)$,其中 $\forall$ 是全称量词。
\end{definition}

\begin{definition}[存在量化]\label{def:存在量化}
    $P(x)$ 的存在量化为语句“论域内存在一个 $x$ 满足 $P(x)$”,记作 $\exists x P(x)$,其中 $\exists$ 是存在量词。
\end{definition}

\begin{collections}
    \begin{example}
        令 $P(x)$ 表示语句“$x + 1 > x$”,论域为全体实数集,求 $\forall x P(x)$ 的真值。
    \end{example}
    \begin{solution}
        显然 $P(x)$ 对所有实数均为真,所以 $\forall x P(x)$ 的值为真。
    \end{solution}

    \spare

    \begin{example}
        令 $P(x)$ 表示语句 “$x = 2$”,论域为 $x < 10$,求 $\exists x P(x)$ 的真值。
    \end{example}
    \begin{solution}
        在论域 $x < 10$ 中存在 $x = 2$ 使得 $P(x)$ 成立,所以 $\exists x P(x)$ 的值为真。
    \end{solution}
\end{collections}

我们也会考虑量化的否定情况。
\begin{theorem}[量化的德·摩根律]
    \begin{equation*}
        \begin{gathered}
            \lnot \forall x P(x) \iff \exists x \lnot P(x) \\
            \lnot \exists x P(x) \iff \forall x \lnot P(x) \\
        \end{gathered}
    \end{equation*}
\end{theorem}

除了以上的两种量化,还存在一个较少使用的量化,唯一性量化。
\begin{definition}[唯一性量化]\label{def:唯一性量化}
    $P(x)$ 的唯一性量化为语句“论域内存在且只存在一个 $x$ 满足 $P(x)$”,记作 $\exists ! x P(x)$,其中 $\exists !$ 是唯一性量词。
\end{definition}

事实上唯一性量化的使用是可以被避免的。
\begin{theorem}[唯一性量化的等价形式]
    唯一性量化可以等价为变量存在性与唯一性的合取。
    \begin{equation*}
        \exists ! x P(x) \iff \exists x P(x) \land \forall x \forall y(P(x) \land P(y) \to (x = y))
    \end{equation*}
\end{theorem}

当论域是有限的时候,也就是论域内的所有所有元素都可以一一列出时,量化语句就可以使用命题逻辑来表述。

\begin{definition}[有限论域上的量化]\label{def:有限域上的量化}
    当论域内元素为 $x_1, x_2, \cdots, x_n$ 时
    \begin{equation*}
        \begin{gathered}
            \forall x P(x) \iff P(x_1) \land P(x_2) \land \cdots \land P(x_n) \\
            \exists x P(x) \iff P(x_1) \lor P(x_2) \lor \cdots \lor P(x_n) \\
        \end{gathered}
    \end{equation*}
\end{definition}

我们还可以将论域限制直接放在量化语句内。
\begin{collections}
    \begin{example}
        用自然语言表述以下量化语句。
        \begin{enumerate}
            \item $\forall x \neq 0(x^2 > 0)$
            \item $\exists y < 3(y^3 = 8)$
        \end{enumerate}
    \end{example}

    \begin{solution}
        \begin{enumerate}
            \item 对于所有满足 $x \neq 0$ 的数 $x$ 有 $x^2 > 0$。
            \item 至少存在一个满足 $y < 3$ 的数 $y$ 有 $y^3 = 8$。
        \end{enumerate}
    \end{solution}
\end{collections}

如果一个变量被量词修饰了,那么这个变量被称为\emph{约束的},反之,该变量则被称为\emph{自由的}。如果量化语句中不存在自由的变量,则该语句才能被转换为命题。
\begin{collections}
    \begin{example}
        请指出以下量化语句中那些变量是约束的,那些变量是自由的。
        \begin{enumerate}
            \item $\forall x P(x)$
            \item $\exists y P(x, y)$
            \item $\forall x (x + y = 1)$
            \item $\exists x P(x) \land Q(x)$
        \end{enumerate}
    \end{example}

    \begin{solution}
        \begin{enumerate}
            \item $P(x)$ 中变量 $x$ 是约束的,没有自由的变量。
            \item $P(x, y)$ 中变量 $y$ 是约束的,变量 $x$ 是自由的。
            \item $x + y = 1$ 中变量 $x$ 是约束的,变量 $y$ 是自由的。
            \item $P(x)$ 中变量 $x$ 是约束的,但 $Q(x)$ 中的变量 $x$ 是自由的,因为 $\exists x P(x) \land Q(x) \iff (\exists x P(x)) \land Q(x)$
        \end{enumerate}
    \end{solution}
\end{collections}

一个量词可以出现在另一个量词的作用域内,例如 $\forall x \forall y P(x,y)$,这样的形式我们称为\emph{嵌套量词}。对于嵌套量词来说,量词的顺序是很重要的,除非全部量词均为全称量词或均为存在量词。

我们来考虑嵌套量词的否定情况。
\begin{collections}
    \begin{example}
        将以下量化语句转化为量词前无否定词的形式。
        \begin{enumerate}
            \item $\lnot \forall x \forall y P(x,y)$
            \item $\lnot \forall x \exists y P(x,y)$
            \item $\lnot \exists x \forall y P(x,y)$
            \item $\lnot \exists x \exists y P(x,y)$
        \end{enumerate}
    \end{example}

    \begin{solution}
        \begin{enumerate}
            \item $\lnot \forall x \forall y P(x,y) \iff \exists x \exists y \lnot P(x,y)$
            \item $\lnot \forall x \exists y P(x,y) \iff \exists x \forall y \lnot P(x,y)$
            \item $\lnot \exists x \forall y P(x,y) \iff \forall x \exists y \lnot P(x,y)$
            \item $\lnot \exists x \exists y P(x,y) \iff \forall x \forall y \lnot P(x,y)$
        \end{enumerate}
    \end{solution}
\end{collections}