\section{命题逻辑}
\begin{introduction}
    \item 命题
    \item 逻辑运算符
    \item 原子命题与复合命题
    \item 逻辑等价
\end{introduction}

\subsection{命题}
命题是命题逻辑中的基本构件,我们在下面给出命题的定义。
\begin{definition}[命题的定义]\label{def:命题的定义}
    命题是一个陈述语句,他可能是真也可能是假,但不可能同时真并且假。语句的真假被称为该语句的真值。如果这个命题的真值为真(记作 T),那么该语句称为真命题,反之,如果这个命题的真值为假(记作 F),那么该语句称为假命题。
\end{definition}

\begin{collections}
    \begin{example}
        下面的语句中哪些是命题,哪些不是命题,如果该语句是命题,请判断它是真命题还是假命题。
        \begin{enumerate}
            \item 上海在东八区。
            \item 北京是中国的首都。
            \item 纽约是中国的城市。
            \item $1 + 1 = 2$
            \item $x + 1 = 2$
            \item 我是谁?
            \item 请帮我把水杯拿过来。
        \end{enumerate}
    \end{example}
    \begin{solution}
        语句 1、2、3、4 是命题,因为他们是陈述句且能确定其真值;语句 5 不是命题,因为尽管它是陈述句,但无法确定其真值;语句 6、7 不是命题,因为它们不是陈述句。在语句 1、2、3、4 中,语句 1、2、4 是真命题,语句 3 是假命题。
    \end{solution}
\end{collections}

命题可以使用字母来表示,这样的字母被称为命题变量,习惯上常用字母 $p, q, r, s, \cdots$ 以及大写拉丁字母。

\subsection{逻辑运算符}
如果我们想从已有的命题产生新的命题,我们就需要逻辑运算符,我们将介绍 3 种逻辑运算符以及 2 种条件运算符。

\subsubsection{否定}
\begin{definition}[命题的否定]\label{def:命题的否定}
    令 $p$ 为命题,命题 $p$ 的否定形式记作 $\lnot p$,读作“非 $p$”,其真值与命题 $p$ 相反。
\end{definition}

在命题逻辑中,我们可以使用真值表来列举命题的所有可能及其真值,$p$ 与 $\lnot p$ 的真值如表 \ref{tab:否定的真值表} 所示。
\begin{table}[H]
    \centering
    \begin{tabular}{c|c}
        \toprule
        \makebox[1cm][c]{$p$} & \makebox[1cm][c]{$\lnot p$} \\
        \midrule
        T & F \\
        F & T \\
        \bottomrule
    \end{tabular}
    \caption{$p$ 与 $\lnot p$ 的真值表}
    \label{tab:否定的真值表}
\end{table}

\subsubsection{合取}
\begin{definition}[命题的合取]\label{def:命题的合取}
    令 $p$ 与 $q$ 为命题,$p$ 与 $q$ 的合取记作 $p \land q$,读作“$p$ 并且 $q$”。当且仅当 $p$ 与 $q$ 均为真时,$p \land q$ 为真,否则为假。
\end{definition}

$p, q$ 与 $p \land q$ 的真值表如表 \ref{tab:合取的真值表} 所示。
\begin{table}[H]
    \centering
    \begin{tabular}{cc|c}
        \toprule
        \makebox[1cm][c]{$p$} & \makebox[1cm][c]{$q$} & \makebox[1cm][c]{$p \land q$} \\
        \midrule
        T & T & T \\
        T & F & F \\
        F & T & F \\
        F & F & F \\
        \bottomrule
    \end{tabular}
    \caption{$p, q$ 与 $p \land q$ 的真值表}
    \label{tab:合取的真值表}
\end{table}

我们可以用符号
\begin{equation*}
    \bigwedge_{i=1}^{n} p_i = p_1 \land p_2 \land \cdots \land p_n
\end{equation*}
来表示一组命题 $p_1,p_2,\cdots,p_n$ 的合取。

\subsubsection{析取}
\begin{definition}[命题的析取]\label{def:命题的析取}
    令 $p$ 与 $q$ 为命题,$p$ 与 $q$ 的析取记作 $p \lor q$,读作“$p$ 或 $q$”。当且仅当 $p$ 与 $q$ 均为假时,$p \lor q$ 为假,否则为真。
\end{definition}

$p, q$ 与 $p \lor q$ 的真值表如表 \ref{tab:析取的真值表} 所示。
\begin{table}[H]
    \centering
    \begin{tabular}{cc|c}
        \toprule
        \makebox[1cm][c]{$p$} & \makebox[1cm][c]{$q$} & \makebox[1cm][c]{$p \lor q$} \\
        \midrule
        T & T & T \\
        T & F & T \\
        F & T & T \\
        F & F & F \\
        \bottomrule
    \end{tabular}
    \caption{$p, q$ 与 $p \lor q$ 的真值表}
    \label{tab:析取的真值表}
\end{table}

与合取类似,我们可以用符号
\begin{equation*}
    \bigvee_{i=1}^{n} p_i = p_1 \lor p_2 \lor \cdots \lor p_n
\end{equation*}
来表示一组命题 $p_1,p_2,\cdots,p_n$ 的析取。

\subsubsection{蕴含}
\begin{definition}[命题的蕴含]\label{def:命题的蕴含}
    令 $p$ 与 $q$ 为命题,$p$ 蕴含 $q$ 记为 $p \to q$,读作“如果 $p$,则 $q$”,当且仅当 $p$ 为真但 $q$ 为假时,$p \to q$ 为假,否则为真。在 $p \to q$ 中,$p$ 称为假设,$q$ 称为结论,因此该命题也被称为条件命题。
\end{definition}

$p, q$ 与 $p \to q$ 的真值表如表 \ref{tab:蕴含的真值表} 所示。
\begin{table}[H]
    \centering
    \begin{tabular}{cc|c}
        \toprule
        \makebox[1cm][c]{$p$} & \makebox[1cm][c]{$q$} & \makebox[1cm][c]{$p \to q$} \\
        \midrule
        T & T & T \\
        T & F & F \\
        F & T & T \\
        F & F & T \\
        \bottomrule
    \end{tabular}
    \caption{$p, q$ 与 $p \to q$ 的真值表}
    \label{tab:蕴含的真值表}
\end{table}

\subsubsection{等价}
\begin{definition}[命题的等价]\label{def:命题的等价}
    令 $p$ 与 $q$ 为命题,$p$ 等价 $q$ 记为 $p \leftrightarrow q$,读作“$p$ 当且仅当 $q$”,当 $p$ 与 $q$ 拥有同样的真值时,$p \leftrightarrow q$ 为真,否则为假。
\end{definition}

$p, q$ 与 $p \leftrightarrow q$ 的真值表如表 \ref{tab:等价的真值表} 所示。
\begin{table}[H]
    \centering
    \begin{tabular}{cc|c}
        \toprule
        \makebox[1cm][c]{$p$} & \makebox[1cm][c]{$q$} & \makebox[1cm][c]{$p \leftrightarrow q$} \\
        \midrule
        T & T & T \\
        T & F & F \\
        F & T & F \\
        F & F & T \\
        \bottomrule
    \end{tabular}
    \caption{$p, q$ 与 $p \leftrightarrow q$ 的真值表}
    \label{tab:等价的真值表}
\end{table}

\subsection{原子命题与复合命题}
在逻辑证明中,不能用简单的命题表示的命题叫做\emph{原子命题},而由已知命题通过逻辑运算符组合而成的命题(如 $p \lor q$)叫做\emph{复合命题}。

在命题演算中,我们需要遵守逻辑运算符的优先级,每个符号的优先级如图 \ref{fig:逻辑运算符的优先级} 表示。

\begin{figure}[H]
    \centering
    \begin{tabular}{c|c}
        \toprule
        \makebox[1cm][c]{运算符}& \makebox[1cm][c]{优先级} \\
        \midrule
        $\lnot$           & 1 \\
        $\land$           & 2 \\
        $\lor$            & 3 \\
        $\to$             & 4 \\
        $\leftrightarrow$ & 5 \\
        \bottomrule
    \end{tabular}
    \caption{逻辑运算符的优先级}
    \label{fig:逻辑运算符的优先级}
\end{figure}

下面我们将通过一个例题来研究如何构造一个复合命题的真值表。
\begin{collections}
    \begin{example}
        试构造命题 $(\lnot p \land q) \to (p \lor \lnot q)$ 的真值表
    \end{example}
    \begin{solution}
        \begin{center}
            \begin{tabular}{cc|cc|c|c|c}
                \toprule
                \makebox[1cm][c]{$p$} & \makebox[1cm][c]{$q$} & \makebox[1cm][c]{$\lnot p$} & \makebox[1cm][c]{$\lnot q$} & \makebox[1.5cm][c]{$\lnot p \land q$} & \makebox[1.5cm][c]{$p \lor \lnot q$} & \makebox[3.5cm][c]{$(\lnot p \land q) \to (p \lor \lnot q)$} \\
                \midrule
                T & T & F & F & F & T & T \\
                T & F & F & T & F & T & T \\
                F & T & T & F & T & F & F \\
                F & F & T & T & F & T & T \\
                \bottomrule
            \end{tabular}
        \end{center}
    \end{solution}
\end{collections}

由此,我们可以对通过真值的角度来复合命题进行分类。
\begin{definition}[重言式、矛盾式、可能式]\label{def:重言式、矛盾式、可能式}
    一个真值永远是真的复合命题被称为重言式(也被称为永真式),一个真值永远是假的复合命题被称为矛盾式(也被称为永假式),既不是重言式也不是矛盾式的复合命题被称为可能式。
\end{definition}

证明一个复合命题是否为重言式(或矛盾式)最简单的方法就是使用真值表。
\begin{collections}
    \begin{example}
        试证明复合命题 $p \lor \lnot p$ 是重言式。
    \end{example}
    \begin{solution}
        构建复合命题 $p \lor \lnot p$ 的真值表
        \begin{center}
            \begin{tabular}{cc|c}
                \toprule
                \makebox[1cm][c]{$p$} & \makebox[1cm][c]{$\lnot p$} & \makebox[2cm][c]{$p \lor \lnot p$} \\
                \midrule
                T & F & T \\
                F & T & T \\
                \bottomrule
            \end{tabular}
        \end{center}

        由真值表可知,复合命题 $p \lor \lnot p$ 的真值总为真,因此复合命题 $p \lor \lnot p$ 是重言式。\qed
    \end{solution}

    \spare

    \begin{example}
        试证明复合命题 $p \land \lnot p$ 是矛盾式。
    \end{example}
    \begin{solution}
        构建复合命题 $p \land \lnot p$ 的真值表
        \begin{center}
            \begin{tabular}{cc|c}
                \toprule
                \makebox[1cm][c]{$p$} & \makebox[1cm][c]{$\lnot p$} & \makebox[2cm][c]{$p \land \lnot p$} \\
                \midrule
                T & F & F \\
                F & T & F \\
                \bottomrule
            \end{tabular}
        \end{center}

        由真值表可知,复合命题 $p \land \lnot p$ 的真值总为假,因此复合命题 $p \land \lnot p$ 是矛盾式。\qed
    \end{solution}
\end{collections}

\subsection{逻辑等价}
在所有可能的情况下,如果两个命题拥有相同的真值,那么我们称这两个命题是逻辑等价的。
\begin{definition}[逻辑等价]\label{def:逻辑等价}
    令 $p$ 与 $q$ 为命题,如果 $p \leftrightarrow q$ 是重言式,那么 $p$ 和 $q$ 被称为是逻辑等价的,记作 $p \iff q$(或 $p \equiv q$)。
\end{definition}

等价关系满足以下三个性质:
\begin{description}
    \item[自反性] 对任意命题 $p$,有 $p \iff p$
    \item[对称性] 对任意命题 $p, q$,如果 $p \iff q$,则 $q \iff p$
    \item[可传递性] 对任意命题 $p, q, r$,如果 $p \iff q$,$q \iff r$,则 $p \iff r$
\end{description}

证明两个命题逻辑等价的方法依然是通过真值表。

\begin{collections}
    \begin{example}
        证明条件命题 $p \to q$ 与其逆否命题 $\lnot q \to \lnot p$ 逻辑等价。
    \end{example}
    \begin{solution}
        构建 $p \to q$ 和 $\lnot q \to \lnot p$ 的真值表
        \begin{center}
            \begin{tabular}{cc|cc|cc}
                \toprule
                \makebox[1cm][c]{$p$} & \makebox[1cm][c]{$q$} & \makebox[1cm][c]{$\lnot p$} & \makebox[1cm][c]{$\lnot q$} & \makebox[1cm][c]{$p \to q$} & \makebox[1.5cm][c]{$\lnot q \to \lnot p$} \\
                \midrule
                T & T & F & F & T & T \\
                T & F & F & T & F & F \\
                F & T & T & F & T & T \\
                F & F & T & T & T & T \\
                \bottomrule
            \end{tabular}
        \end{center}

        由真值表可知,$p \to q$ 与 $\lnot q \to \lnot p$ 有着相同的真值,即 $p \to q \iff \lnot q \to \lnot p$。\qed
    \end{solution}
\end{collections}

下面我们将给出一些常见命题的等价形式,限于篇幅我们不可能将证明一一列出,有兴趣的同学可以自行证明。

\begin{longtable}{c|c}
    \toprule
    等价式 & \makebox[3cm][c]{名称} \\
    \midrule
    \endhead
    $p \land q \iff q \land p$ & \multirow{2}{*}{交换律} \\
    $p \lor q \iff q \lor p$ & \\
    \midrule
    $(p \land q) \land r \iff p \land (q \land r)$ & \multirow{2}{*}{结合律} \\
    $(p \lor q) \lor r \iff p \lor (q \lor r)$ & \\
    \midrule
    $p \land (q \lor r) \iff (p \land q) \lor (p \land r)$ & \multirow{2}{*}{分配律} \\
    $p \lor (q \land r) \iff (p \lor q) \land (p \lor r)$ & \\
    \midrule
    $p \land p \iff p$ & \multirow{2}{*}{幂等律} \\
    $p \lor p \iff p$ & \\
    \midrule
    $p \land \symbf{T} \iff p$ & \multirow{2}{*}{恒等律} \\
    $p \lor \symbf{F} \iff p$ & \\
    \midrule
    $p \lor \symbf{T} \iff \symbf{T}$ & \multirow{2}{*}{支配律} \\
    $p \land \symbf{F} \iff \symbf{F}$ & \\
    \midrule
    $p \lor \lnot p \iff \symbf{T}$ & \multirow{2}{*}{否定率} \\
    $p \land \lnot p \iff \symbf{F}$ & \\
    \midrule
    $\lnot(\lnot p) \iff p$ & 双重否定律 \\
    \midrule
    $\lnot(p \land q) \iff \lnot p \lor \lnot q$ & \multirow{2}{*}{德·摩根律} \\
    $\lnot(p \lor q) \iff \lnot p \land \lnot q$ & \\
    \midrule
    $p \to q \iff \lnot p \lor q$ & \multirow{2}{*}{条件运算的等价} \\
    $p \leftrightarrow q \iff (p \to q) \land (q \to p)$ & \\
    \bottomrule
\end{longtable}
