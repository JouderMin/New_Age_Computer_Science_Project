\section{命题}
\begin{introduction}
    \item 命题
    \item 逻辑运算符
    \item 复合命题
    \item 逻辑运算符的优先级
\end{introduction}

\subsection{命题}
命题是命题逻辑中的基本构件,我们在下面给出命题的定义。
\begin{definition}[命题的定义]\label{def:命题的定义}
    命题是一个陈述语句,他可能是真也可能是假,但不可能同时真并且假。语句的真假被称为该语句的真值。如果这个命题的真值为真(记作 T),那么该语句称为真命题,反之,如果这个命题的真值为假(记作 F),那么该语句称为假命题。
\end{definition}

\begin{collections}
    \begin{example}
        下面的语句中哪些是命题,哪些不是命题,如果该语句是命题,请判断它是真命题还是假命题。
        \begin{enumerate}
            \item 上海在东八区。
            \item 北京是中国的首都。
            \item 纽约是中国的城市。
            \item $1 + 1 = 2$。
            \item $x + 1 = 2$。
            \item 我是谁?
            \item 请帮我把水杯拿过来。
        \end{enumerate}
    \end{example}
    \begin{solution}
        语句 1、2、3、4 是命题,因为他们是陈述句且能确定其真值;语句 5 不是命题,因为尽管它是陈述句,但无法确定其真值;语句 6、7 不是命题,因为它们不是陈述句。在语句 1、2、3、4 中,语句 1、2、4 是真命题,语句 3 是假命题。
    \end{solution}
\end{collections}

命题可以使用字母来表示,这样的字母被称为命题变量,习惯上常用字母 $p, q, r, s, \cdots$ 以及大写拉丁字母。

\section{逻辑运算符}
如果我们想从已有的命题产生新的命题,我们就需要逻辑运算符,我们将介绍 3 种基本逻辑运算符以及 3 种常用的逻辑运算符。

\subsection{否定}
\begin{definition}[命题的否定]\label{def:命题的否定}
    令 $p$ 为命题,命题 $p$ 的否定形式记作 $\lnot p$,读作“非 $p$”,其真值与命题 $p$ 相反。
\end{definition}

在命题逻辑中,我们可以使用真值表来列举命题的所有可能及其真值,$p$ 与 $\lnot p$ 的真值如表 \ref{tab:否定的真值表} 所示。
\begin{table}[H]
    \centering
    \begin{tabular}{c|c}
        \toprule
        \makebox[1cm][c]{$p$} & \makebox[1cm][c]{$\lnot p$} \\
        \midrule
        T & F \\
        F & T \\
        \bottomrule
    \end{tabular}
    \caption{$p$ 与 $\lnot p$ 的真值表}
    \label{tab:否定的真值表}
\end{table}

\subsection{合取}
\begin{definition}
    令 $p$ 与 $q$ 为命题,$p$ 与 $q$ 的合取记作 $p \land q$,读作“$p$ 并且 $q$”。当且仅当 $p$ 与 $q$ 均为真时,$p \land q$ 为真,否则为假。
\end{definition}
$p, q$ 与 $p \land q$ 的真值表如表 \ref{tab:合取的真值表} 所示。

\begin{table}[H]
    \centering
    \begin{tabular}{cc|c}
        \toprule
        \makebox[1cm][c]{$p$} & \makebox[1cm][c]{$q$} & \makebox[1cm][c]{$p \land q$} \\
        \midrule
        T & T & T \\
        T & F & F \\
        F & T & F \\
        F & F & F \\
        \bottomrule
    \end{tabular}
    \caption{$p, q$ 与 $p \land q$ 的真值表}
    \label{tab:合取的真值表}
\end{table}
