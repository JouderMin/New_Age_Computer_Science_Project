\section{图的表示}
\begin{introduction}
    \item 邻接表
    \item 邻接矩阵
    \item 关联矩阵
\end{introduction}

\subsection{邻接表}
邻接表可以用来表示不存在多重边的图。
\begin{collections}
    \begin{example}
        用邻接表表示以下无向图。
        \vspace{-2em}
        \begin{center}
            \begin{tikzpicture}
                \GraphInit[vstyle=normal]
                \SetGraphUnit{1.5}\SetVertexMath
                \Vertex{a}
                \EA(a){b}
                \SO(a){c}
                \EA(c){d}
                \EA(d){e}
                \Edge[style={-}](a)(b)
                \Edge[style={-}](a)(c)
                \Edge[style={-}](a)(d)
                \Edge[style={-}](b)(d)
                \Edge[style={-}](b)(e)
                \Edge[style={-}](c)(d)
                \Edge[style={-}](d)(e)
                \Loop[style={-},dist=1.5cm](a)
            \end{tikzpicture}
        \end{center}
    \end{example}
    \begin{solution}
        \begin{center}
            \begin{tabular}{c|c}
                \toprule
                \makebox[2cm][c]{顶点} & \makebox[2cm][c]{邻居} \\
                \midrule
                $a$ & $a, b, c, d$ \\
                $b$ & $a, d, e$ \\
                $c$ & $a, d$ \\
                $d$ & $a, b, c, e$ \\
                $e$ & $b, d$\\
                \bottomrule
            \end{tabular}
        \end{center}
    \end{solution}

    \spare

    \begin{example}
        用邻接表表示以下有向图。
        \vspace{-2em}
        \begin{center}
            \begin{tikzpicture}
                \GraphInit[vstyle=normal]
                \SetGraphUnit{1.5}\SetVertexMath
                \Vertex{a}
                \EA(a){b}
                \SO(a){c}
                \EA(c){d}
                \EA(d){e}
                \Edge[style={->, bend left}](a)(b)
                \Edge[style={->, bend left}](b)(a)
                \Edge[style={->}](a)(c)
                \Edge[style={->}](a)(d)
                \Edge[style={->}](b)(d)
                \Edge[style={->}](b)(e)
                \Edge[style={->}](c)(d)
                \Edge[style={->}](d)(e)
                \Loop[style={->},dist=1.5cm](a)
                \Loop[style={->},dist=1.5cm, dir=EA](e)
            \end{tikzpicture}
            \vspace{-2em}
        \end{center}
    \end{example}
    \begin{solution}
        \begin{center}
            \begin{tabular}{c|c}
                \toprule
                \makebox[2cm][c]{起点} & \makebox[2cm][c]{终点} \\
                \midrule
                $a$ & $a, b, c, d$ \\
                $b$ & $a, d, e$ \\
                $c$ & $d$ \\
                $d$ & $e$ \\
                $e$ & $e$ \\
                \bottomrule
            \end{tabular}
        \end{center}
    \end{solution}
\end{collections}

\subsection{邻接矩阵}
当图中边的数量较多时,可以使用邻接矩阵来表示图。

无向图 $G=(V, E)$ 的邻接矩阵是一个大小为 $|V| \times |V|$ 的矩阵,矩阵中的元素 $a_{ij}$ 表示顶点 $v_i$ 与 $v_j$ 之间边的数量。无向图的邻接矩阵是关于其主对角线对称的。

有向图 $G=(V, E)$ 的邻接矩阵也是一个大小为 $|V| \times |V|$ 的矩阵,不过矩阵中的元素 $a_{ij}$ 表示以顶点 $v_i$ 为起点,$v_j$ 为终点的边的数量。有向图的邻接矩阵不一定是对称的。
\begin{collections}
    \begin{example}
        用邻接矩阵表示以下无向图。
        \vspace{-2em}
        \begin{center}
            \begin{tikzpicture}
                \GraphInit[vstyle=normal]
                \SetGraphUnit{1.5}\SetVertexMath
                \Vertex{a}
                \EA(a){b}
                \SO(a){c}
                \EA(c){d}
                \EA(d){e}
                \Edge[style={-}](a)(b)
                \Edge[style={-}](a)(c)
                \Edge[style={-}](a)(d)
                \Edge[style={-}](b)(d)
                \Edge[style={-}](b)(e)
                \Edge[style={-}](c)(d)
                \Edge[style={-}](d)(e)
                \Loop[style={-},dist=1.5cm](a)
            \end{tikzpicture}
        \end{center}
    \end{example}
    \begin{solution}
        $$
        \begin{bmatrix}
            1 & 1 & 1 & 1 & 0 \\
            1 & 0 & 0 & 1 & 1 \\
            1 & 0 & 0 & 1 & 0 \\
            1 & 1 & 1 & 0 & 1 \\
            0 & 1 & 0 & 1 & 0 \\
        \end{bmatrix}
        $$
    \end{solution}

    \spare

    \begin{example}
        用邻接矩阵表示以下有向图。
        \vspace{-2em}
        \begin{center}
            \begin{tikzpicture}
                \GraphInit[vstyle=normal]
                \SetGraphUnit{1.5}\SetVertexMath
                \Vertex{a}
                \EA(a){b}
                \SO(a){c}
                \EA(c){d}
                \EA(d){e}
                \Edge[style={->}](a)(b)
                \Edge[style={->, bend left}](a)(b)
                \Edge[style={->, bend left}](b)(a)
                \Edge[style={->}](a)(c)
                \Edge[style={->}](a)(d)
                \Edge[style={->}](b)(d)
                \Edge[style={->}](b)(e)
                \Edge[style={->}](c)(d)
                \Edge[style={->}](d)(e)
                \Loop[style={->},dist=1.5cm](a)
                \Loop[style={->},dist=1.5cm, dir=EA](e)
            \end{tikzpicture}
            \vspace{-2em}
        \end{center}
    \end{example}
    \begin{solution}
        $$
        \begin{bmatrix}
            1 & 2 & 1 & 1 & 0 \\
            1 & 0 & 0 & 1 & 1 \\
            0 & 0 & 0 & 1 & 0 \\
            0 & 0 & 0 & 0 & 1 \\
            0 & 0 & 0 & 0 & 1 \\
        \end{bmatrix}
        $$
    \end{solution}
\end{collections}

可以看出,对于无向图的邻接矩阵,存在
\begin{equation*}
    \sum_{j=1}^{|V|}(a_{ij} + a_{ji})=\symrm{deg}(v_i)
\end{equation*}
对于有向图的邻接矩阵,存在
\begin{equation*}
    \begin{aligned}
        \sum_{j=1}^{|V|}a_{ij}=\symrm{deg}^+(v_i) \\
        \sum_{j=1}^{|V|}a_{ji}=\symrm{deg}^-(v_i) \\
    \end{aligned}
\end{equation*}

如果图中不存在多重边,则易知该图的邻接矩阵是一个 01 矩阵。
\begin{info}{01 矩阵}
如果一个矩阵中的任意元素 $a_{ij}$ 均存在 $a_{ij}=0 \lor a_{ij}=1$,则该矩阵被称为 01 矩阵。
\end{info}

\subsection{关联矩阵}
关联矩阵于邻接矩阵不同,关联矩阵只能用于表示无向图。设无向图 $G=(V,E)$,其关联矩阵是一个大小为 $|V|\times|E|$ 的 01 矩阵,矩阵中的元素 $a_{ij}$ 有以下特点。
\begin{equation*}
    a_{ij}=
    \begin{cases}
        1 & \text{$v_i$ 与 $e_j$ 相关联} \\
        0 & \text{$v_i$ 与 $e_j$ 无关} \\
    \end{cases}
\end{equation*}

\begin{collections}
    \begin{example}
        用关联矩阵表示以下无向图。
        \vspace{-2em}
        \begin{center}
            \begin{tikzpicture}
                \GraphInit[vstyle=normal]
                \SetGraphUnit{1.5}\SetVertexMath
                \Vertex{a}
                \EA(a){b}
                \SO(a){c}
                \EA(c){d}
                \EA(d){e}
                \Edge[style={-}, label={$v_1$}](a)(b)
                \Edge[style={-}, label={$v_2$}](a)(c)
                \Edge[style={-}, label={$v_3$}](a)(d)
                \Edge[style={-}, label={$v_4$}](b)(d)
                \Edge[style={-}, label={$v_5$}](b)(e)
                \Edge[style={-}, label={$v_6$}](c)(d)
                \Edge[style={-}, label={$v_7$}](d)(e)
                \Loop[style={-}, labelstyle={fill=white}, label={$v_8$}, dist=1.5cm](a)
            \end{tikzpicture}
        \end{center}
    \end{example}
        \begin{solution}
            $$
            \begin{bmatrix}
                1 & 1 & 1 & 0 & 0 & 0 & 0 & 1 \\
                1 & 0 & 0 & 1 & 1 & 0 & 0 & 0 \\
                0 & 1 & 0 & 0 & 0 & 1 & 1 & 0 \\
                0 & 0 & 1 & 1 & 0 & 1 & 0 & 0 \\
                0 & 0 & 0 & 0 & 1 & 0 & 1 & 0 \\
            \end{bmatrix}
            $$
        \end{solution}
\end{collections}

\newpage