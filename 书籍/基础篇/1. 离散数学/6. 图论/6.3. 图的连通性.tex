\section{图的连通性}
\begin{introduction}
    \item 通路与回路
    \item 无向图的连通性
    \item 有向图的连通性
\end{introduction}

\subsection{通路与回路}

\begin{definition}[无向图的通路与回路]\label{def:无向图的通路与回路}
    设 $n$ 是正整数且 $G=(V,E)$ 是无向图,$u,v \in V$。在 $G$ 中从 $u$ 到 $v$ 的长度为 $n$ 的通路指 $G$ 中边的序列 $(e_1,e_2,\cdots,e_n)$,其中边 $e_1$ 与顶点 $u$ 相关联,边 $e_n$ 与顶点 $v$ 相关联,根据这个边的序列,可以列出与每条边相关联的顶点序列 $(u,\cdots,v)$,当 $G$ 是简单无向图时,可以使用该顶点顶点序列来表示这条通路。如果一条通路在相同的顶点上开始和结束且该通路长度大于 $0$,则称这条通路是一条回路。若通路或回路中不包含相同的边,则称他是简单的。
\end{definition}

\begin{collections}
    \begin{example}
        请指出下图中任意一条回路和从顶点 $a$ 到 $e$ 的任意一条通路。
        \begin{center}
            \begin{tikzpicture}
                \GraphInit[vstyle=normal]
                \SetGraphUnit{1.5}\SetVertexMath
                \Vertex{a}
                \EA(a){b}
                \SO(a){c}
                \EA(c){d}
                \EA(d){e}
                \Edge[style={-}](a)(b)
                \Edge[style={-}](a)(c)
                \Edge[style={-}](a)(d)
                \Edge[style={-}](b)(d)
                \Edge[style={-}](b)(e)
                \Edge[style={-}](c)(d)
                \Edge[style={-}](d)(e)
            \end{tikzpicture}
        \end{center}
    \end{example}
    \begin{solution}
        上图中存在回路 $(a, b, d, a)$ 及从顶点 $a$ 到 $e$ 的通路 $(a, d, e)$。
        \begin{center}
            \begin{tikzpicture}
                \SetGraphUnit{1.5}\SetVertexMath
                \Vertex{a}
                \EA(a){b}
                \SO(a){c}
                \EA(c){d}
                \EA(d){e}
                \Edge[style={-}, color=third](a)(b)
                \Edge[style={-}](a)(c)
                \Edge[style={-}, color=third](a)(d)
                \Edge[style={-}, color=third](b)(d)
                \Edge[style={-}](b)(e)
                \Edge[style={-}](c)(d)
                \Edge[style={-}](d)(e)
            \end{tikzpicture}
            \hspace{3em}
            \begin{tikzpicture}
                \SetGraphUnit{1.5}\SetVertexMath
                \Vertex{a}
                \EA(a){b}
                \SO(a){c}
                \EA(c){d}
                \EA(d){e}
                \Edge[style={-}](a)(b)
                \Edge[style={-}](a)(c)
                \Edge[style={-}, color=third](a)(d)
                \Edge[style={-}](b)(d)
                \Edge[style={-}](b)(e)
                \Edge[style={-}](c)(d)
                \Edge[style={-}, color=third](d)(e)
            \end{tikzpicture}
        \end{center}
    \end{solution}
\end{collections}

由定义 \ref{def:无向图的通路与回路} 我们可以类比作出有向图中通路与回路的定义。

\begin{definition}[有向图的通路与回路]\label{def:有向图的通路与回路}
    设 $n$ 是正整数且 $G=(V,E)$ 是有向图,$x_1,x_2,x_{n-1},x_n \in V$。在 $G$ 中从 $x_1$ 到 $x_n$ 的长度为 $n$ 的通路指 $G$ 中边的序列 $(e_1,e_2,\cdots,e_n)$,其中边 $e_1 = (x_1,x_2)$,边 $e_n = (x_{n-1},x_n)$,根据这个边的序列,可以列出与每条边相关联的顶点序列 $(x_1,x_2,\cdots,x_n)$,当 $G$ 是简单有向图时,可以使用该顶点顶点序列来表示这条通路。如果一条通路在相同的顶点上开始和结束且该通路长度大于 $0$,则称这条通路是一条回路。若通路或回路中不包含相同的边,则称他是简单的。
\end{definition}

\begin{collections}
    \begin{example}
        请指出下图中任意一条回路和从顶点 $a$ 到 $e$ 的任意一条通路。
        \begin{center}
            \begin{tikzpicture}
                \GraphInit[vstyle=normal]
                \SetGraphUnit{1.5}\SetVertexMath
                \Vertex{a}
                \EA(a){b}
                \SO(a){c}
                \EA(c){d}
                \EA(d){e}
                \Edge[style={->}](a)(b)
                \Edge[style={->}](a)(c)
                \Edge[style={->}](d)(a)
                \Edge[style={->}](b)(d)
                \Edge[style={->}](e)(b)
                \Edge[style={->}](c)(d)
                \Edge[style={->}](d)(e)
            \end{tikzpicture}
        \end{center}
    \end{example}
    \begin{solution}
        上图中存在回路 $(b, d, e, b)$ 及从顶点 $a$ 到 $e$ 的通路 $(a, b, d, e)$。
        \begin{center}
            \begin{tikzpicture}
                \SetGraphUnit{1.5}\SetVertexMath
                \Vertex{a}
                \EA(a){b}
                \SO(a){c}
                \EA(c){d}
                \EA(d){e}
                \Edge[style={->}](a)(b)
                \Edge[style={->}](a)(c)
                \Edge[style={->}](d)(a)
                \Edge[style={->}, color=third](b)(d)
                \Edge[style={->}, color=third](e)(b)
                \Edge[style={->}](c)(d)
                \Edge[style={->}, color=third](d)(e)
            \end{tikzpicture}
            \hspace{3em}
            \begin{tikzpicture}
                \SetGraphUnit{1.5}\SetVertexMath
                \Vertex{a}
                \EA(a){b}
                \SO(a){c}
                \EA(c){d}
                \EA(d){e}
                \Edge[style={->}, color=third](a)(b)
                \Edge[style={->}](a)(c)
                \Edge[style={->}](d)(a)
                \Edge[style={->}, color=third](b)(d)
                \Edge[style={->}](e)(b)
                \Edge[style={->}](c)(d)
                \Edge[style={->}, color=third](d)(e)
            \end{tikzpicture}
        \end{center}
    \end{solution}
\end{collections}

\subsection{无向图的连通性}
\begin{definition}[无向图的连通性]\label{def:无向图的连通性}
    若无向图中每一对不同的顶点之间都有通路,则称该图为连通的,否则称为不连通的。
\end{definition}

从定义 \ref{def:无向图的连通性} 中可以推出定理 \ref{thm:无向图定理通路}。

\begin{theorem}\label{thm:无向图定理通路}
    在连通无向图中的每一对不同顶点之间都存在简单通路。
\end{theorem}

\begin{collections}
    \begin{example}
        证明下图是不连通的。
        \begin{center}
            \begin{tikzpicture}
                \GraphInit[vstyle=normal]
                \SetGraphUnit{1.5}\SetVertexMath
                \Vertex{a}
                \EA(a){b}
                \EA(b){c}
                \SO(b){e}
                \SO(a){d}
                \SO(c){f}
                \Edge[style={-}](a)(b)
                \Edge[style={-}](a)(d)
                \Edge[style={-}](b)(d)
                \Edge[style={-}](c)(e)
                \Edge[style={-}](c)(f)
            \end{tikzpicture}
        \end{center}
    \end{example}
    \begin{solution}
        从上图可知顶点 $a$ 与 顶点 $e$ 之间不存在通路,违背了定义 \ref{def:无向图的连通性},因此该图是不连通的。
    \end{solution}
\end{collections}