\section{图的连通性}
\begin{introduction}
    \item 通路与回路
    \item 顶点之间的通路数量
    \item 无向图的连通性
    \item 有向图的连通性
\end{introduction}

\subsection{通路与回路}

\begin{definition}[无向图的通路与回路]\label{def:无向图的通路与回路}
    设 $n$ 是正整数且 $G=(V,E)$ 是无向图,$u,v \in V$。在 $G$ 中从 $u$ 到 $v$ 的长度为 $n$ 的通路指 $G$ 中边的序列 $(e_1,e_2,\cdots,e_n)$,其中边 $e_1$ 与顶点 $u$ 相关联,边 $e_n$ 与顶点 $v$ 相关联,根据这个边的序列,可以列出与每条边相关联的顶点序列 $(u,\cdots,v)$,当 $G$ 是简单无向图时,可以使用该顶点顶点序列来表示这条通路。如果一条通路在相同的顶点上开始和结束且该通路长度大于 $0$,则称这条通路是一条回路。若通路或回路中不包含相同的边,则称他是简单的。
\end{definition}

\begin{collections}
    \begin{example}
        请指出下图中任意一条回路和从顶点 $a$ 到 $e$ 的任意一条通路。
        \begin{center}
            \begin{tikzpicture}
                \GraphInit[vstyle=normal]
                \SetGraphUnit{1.5}\SetVertexMath
                \Vertex{a}
                \EA(a){b}
                \SO(a){c}
                \EA(c){d}
                \EA(d){e}
                \Edge[style={-}](a)(b)
                \Edge[style={-}](a)(c)
                \Edge[style={-}](a)(d)
                \Edge[style={-}](b)(d)
                \Edge[style={-}](b)(e)
                \Edge[style={-}](c)(d)
                \Edge[style={-}](d)(e)
            \end{tikzpicture}
        \end{center}
    \end{example}
    \begin{solution}
        上图中存在回路 $(a, b, d, a)$ 及从顶点 $a$ 到 $e$ 的通路 $(a, d, e)$。
        \begin{center}
            \begin{tikzpicture}
                \SetGraphUnit{1.5}\SetVertexMath
                \Vertex{a}
                \EA(a){b}
                \SO(a){c}
                \EA(c){d}
                \EA(d){e}
                \Edge[style={-}, color=third](a)(b)
                \Edge[style={-}](a)(c)
                \Edge[style={-}, color=third](a)(d)
                \Edge[style={-}, color=third](b)(d)
                \Edge[style={-}](b)(e)
                \Edge[style={-}](c)(d)
                \Edge[style={-}](d)(e)
            \end{tikzpicture}
            \hspace{3em}
            \begin{tikzpicture}
                \SetGraphUnit{1.5}\SetVertexMath
                \Vertex{a}
                \EA(a){b}
                \SO(a){c}
                \EA(c){d}
                \EA(d){e}
                \Edge[style={-}](a)(b)
                \Edge[style={-}](a)(c)
                \Edge[style={-}, color=third](a)(d)
                \Edge[style={-}](b)(d)
                \Edge[style={-}](b)(e)
                \Edge[style={-}](c)(d)
                \Edge[style={-}, color=third](d)(e)
            \end{tikzpicture}
        \end{center}
    \end{solution}
\end{collections}

由定义 \ref{def:无向图的通路与回路} 我们可以类比作出有向图中通路与回路的定义。

\begin{definition}[有向图的通路与回路]\label{def:有向图的通路与回路}
    设 $n$ 是正整数且 $G=(V,E)$ 是有向图,$x_1,x_2,x_{n-1},x_n \in V$。在 $G$ 中从 $x_1$ 到 $x_n$ 的长度为 $n$ 的通路指 $G$ 中边的序列 $(e_1,e_2,\cdots,e_n)$,其中边 $e_1 = (x_1,x_2)$,边 $e_n = (x_{n-1},x_n)$,根据这个边的序列,可以列出与每条边相关联的顶点序列 $(x_1,x_2,\cdots,x_n)$,当 $G$ 是简单有向图时,可以使用该顶点顶点序列来表示这条通路。如果一条通路在相同的顶点上开始和结束且该通路长度大于 $0$,则称这条通路是一条回路。若通路或回路中不包含相同的边,则称他是简单的。
\end{definition}

\begin{collections}
    \begin{example}
        请指出下图中任意一条回路和从顶点 $a$ 到 $e$ 的任意一条通路。
        \begin{center}
            \begin{tikzpicture}
                \GraphInit[vstyle=normal]
                \SetGraphUnit{1.5}\SetVertexMath
                \Vertex{a}
                \EA(a){b}
                \SO(a){c}
                \EA(c){d}
                \EA(d){e}
                \Edge[style={->}](a)(b)
                \Edge[style={->}](a)(c)
                \Edge[style={->}](d)(a)
                \Edge[style={->}](b)(d)
                \Edge[style={->}](e)(b)
                \Edge[style={->}](c)(d)
                \Edge[style={->}](d)(e)
            \end{tikzpicture}
        \end{center}
    \end{example}
    \begin{solution}
        上图中存在回路 $(b, d, e, b)$ 及从顶点 $a$ 到 $e$ 的通路 $(a, b, d, e)$。
        \begin{center}
            \begin{tikzpicture}
                \SetGraphUnit{1.5}\SetVertexMath
                \Vertex{a}
                \EA(a){b}
                \SO(a){c}
                \EA(c){d}
                \EA(d){e}
                \Edge[style={->}](a)(b)
                \Edge[style={->}](a)(c)
                \Edge[style={->}](d)(a)
                \Edge[style={->}, color=third](b)(d)
                \Edge[style={->}, color=third](e)(b)
                \Edge[style={->}](c)(d)
                \Edge[style={->}, color=third](d)(e)
            \end{tikzpicture}
            \hspace{3em}
            \begin{tikzpicture}
                \SetGraphUnit{1.5}\SetVertexMath
                \Vertex{a}
                \EA(a){b}
                \SO(a){c}
                \EA(c){d}
                \EA(d){e}
                \Edge[style={->}, color=third](a)(b)
                \Edge[style={->}](a)(c)
                \Edge[style={->}](d)(a)
                \Edge[style={->}, color=third](b)(d)
                \Edge[style={->}](e)(b)
                \Edge[style={->}](c)(d)
                \Edge[style={->}, color=third](d)(e)
            \end{tikzpicture}
        \end{center}
    \end{solution}
\end{collections}

\subsection{顶点间的通路数量}
我们可以通过图的邻接矩阵来确定顶点间的通路数量。
\begin{theorem}
    设 $G$ 是一个图,矩阵 $\symbf{A}$ 是图 $G$ 的邻接矩阵,$v_1, v_2, \cdots, v_n$ 是图 $G$ 的顶点。从 $v_i$ 到 $v_j$ 长度为 $r\,(r > 0)$ 的通路数量为 $\symbf{A}^r$ 的第 $(i, j)$ 项。
\end{theorem}

\begin{collections}
    \begin{example}
        图 $G$ 如下图所示,求从顶点 $a$ 到 $c$ 长度为 $3$ 的通路数量。

        \begin{center}
            \begin{tikzpicture}
                \SetGraphUnit{1.5}\SetVertexMath
                \Vertex{a}
                \EA(a){b}
                \SO(a){c}
                \EA(c){d}
                \Edge[style={-}](a)(b)
                \Edge[style={-}](a)(d)
                \Edge[style={-}](b)(c)
                \Edge[style={-}](c)(d)
            \end{tikzpicture}
        \end{center}
    \end{example}

    \begin{solution}
        建立图 $G$ 的邻接矩阵 $\symbf{A}$,可知
        \begin{equation*}
            \symbf{A}=
            \begin{bmatrix}
                0 & 1 & 0 & 1 \\
                1 & 0 & 1 & 0 \\
                0 & 1 & 0 & 1 \\
                1 & 0 & 1 & 0 \\
            \end{bmatrix}
        \end{equation*}
        计算,得
        \begin{equation*}
            \symbf{A}^3=
            \begin{bmatrix}
                0 & 4 & 0 & 4 \\
                4 & 0 & 4 & 0 \\
                0 & 4 & 0 & 4 \\
                4 & 0 & 4 & 0 \\
            \end{bmatrix}
        \end{equation*}
        可知 $a_{13} = 0$,所以从顶点 $a$ 到 $c$ 长度为 $3$ 的通路数量为 $0$。
    \end{solution}
\end{collections}


\subsection{无向图的连通性}
\begin{definition}[无向图的连通性]\label{def:无向图的连通性}
    如果无向图中每一对不同的顶点之间都有通路,则称该图为连通的,否则称为不连通的。
\end{definition}

从定义 \ref{def:无向图的连通性} 中可以推出定理 \ref{thm:无向图定理通路}。

\begin{theorem}\label{thm:无向图定理通路}
    在连通无向图中的每一对不同顶点之间都存在简单通路。
\end{theorem}

\begin{collections}
    \begin{example}
        证明下图是不连通的。
        \begin{center}
            \begin{tikzpicture}
                \GraphInit[vstyle=normal]
                \SetGraphUnit{1.5}\SetVertexMath
                \Vertex{a}
                \EA(a){b}
                \EA(b){c}
                \SO(b){e}
                \SO(a){d}
                \SO(c){f}
                \Edge[style={-}](a)(b)
                \Edge[style={-}](a)(d)
                \Edge[style={-}](b)(d)
                \Edge[style={-}](c)(e)
                \Edge[style={-}](c)(f)
            \end{tikzpicture}
        \end{center}
    \end{example}
    \begin{solution}
        从上图可知顶点 $a$ 与 顶点 $e$ 之间不存在通路,因此该图是不连通的。
    \end{solution}
\end{collections}

\subsection{有向图的连通性}
有向图中有两种连通性概念。
\begin{definition}[有向图的强连通性]\label{def:有向图的强连通性}
    如果有向图中的任意顶点 $a$ 和 $b$之间都存在通路 $(a,b)$ 和 $(b,a)$,则该图是强连通的。
\end{definition}
\begin{definition}[有向图的弱连通性]\label{def:有向图的弱连通性}
    如果有向图的基本无向图是连通的,则称该有向图是弱连通的。
\end{definition}
\begin{info}{基本无向图}
    有向图的基本无向图指的是该有向图忽略所有边的方向后所生成的无向图。
    \begin{center}
        \begin{tikzpicture}
            \GraphInit[vstyle=normal]
            \SetGraphUnit{1.5}\SetVertexMath
            \Vertex{a}
            \EA(a){b}
            \SO(a){c}
            \EA(c){d}
            \Edge[style={->}](a)(b)
            \Edge[style={->}](a)(c)
            \Edge[style={->}](a)(d)
            \Edge[style={->}](b)(c)
            \Edge[style={->}](b)(d)
            \Edge[style={->}](c)(d)
        \end{tikzpicture}
        \hspace{3em}
        \begin{tikzpicture}
            \GraphInit[vstyle=normal]
            \SetGraphUnit{1.5}\SetVertexMath
            \Vertex{a}
            \EA(a){b}
            \SO(a){c}
            \EA(c){d}
            \Edge[style={-}](a)(b)
            \Edge[style={-}](a)(c)
            \Edge[style={-}](a)(d)
            \Edge[style={-}](b)(c)
            \Edge[style={-}](b)(d)
            \Edge[style={-}](c)(d)
        \end{tikzpicture}
    \end{center}
\end{info}
