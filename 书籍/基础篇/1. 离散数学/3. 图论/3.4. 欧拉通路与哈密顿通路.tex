\section{欧拉通路与哈密顿通路}
\begin{introduction}
    \item 欧拉通路与欧拉回路
    \item 哈密顿通路与哈密顿回路
\end{introduction}

\subsection{欧拉通路与欧拉回路}
我们先给出欧拉通路与欧拉回路的定义。
\begin{definition}[欧拉通路与欧拉回路]\label{def:欧拉通路与欧拉回路}
    图 $G$ 中的欧拉通路是包含 $G$ 中每一条边的简单通路。图 $G$ 中的欧拉回路是包含图 $G$ 中每一条边的简单回路。
\end{definition}

\begin{collections}
    \begin{example}
        判断下列图中是否存在欧拉回路,若存在,请指出一条欧拉回路,若不存在,则判断其是否存在欧拉通路。
        \begin{center}
            \begin{tikzpicture}
                \SetGraphUnit{1.5}\SetVertexMath
                \Vertex{a}
                \EA(a){b}
                \SO(a){c}
                \EA(c){d}
                \Edge[style={-}](a)(b)
                \Edge[style={-}](a)(d)
                \Edge[style={-}](b)(c)
                \Edge[style={-}](c)(d)
                \coordinate (label) at ($ (c)!0.5!(d) $);
                \node[below=of label, yshift=1.5em]{$G_1$};
            \end{tikzpicture}
            \hspace{3em}
            \begin{tikzpicture}
                \SetGraphUnit{1.5}\SetVertexMath
                \Vertex{a}
                \EA(a){b}
                \SO(a){c}
                \EA(c){d}
                \Edge[style={->}](a)(d)
                \Edge[style={->}](d)(c)
                \Edge[style={->}](c)(b)
                \Edge[style={->}](b)(a)
                \coordinate (label) at ($ (c)!0.5!(d) $);
                \node[below=of label, yshift=1.5em]{$G_2$};
            \end{tikzpicture}
        \end{center}
    \end{example}
\end{collections}