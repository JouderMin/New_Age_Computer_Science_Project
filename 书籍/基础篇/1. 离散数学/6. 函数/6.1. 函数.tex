\section{函数}
\begin{introduction}
    \item 函数的概念
    \item 函数的复合
\end{introduction}

\subsection{函数的概念}
我们先了解函数中的一些定义。
\begin{definition}[函数的定义]\label{def:函数的定义}
    令 $A$ 与 $B$ 为非空集合,存在 $f$ 使得对于集合 $A$ 内的所有元素,只存在一个集合 $B$ 中的元素使得 $f(a) = b$ 成立,也就是说 $\forall a \in A \, \exists! b \in B(f(a)=b)$ 成立,则该 $f$ 称为从 $A$ 到 $B$ 的函数,记作 $f:A \to B$。
\end{definition}

\begin{definition}[像与原像]\label{def:像与原像}
    如果 $f(a) = b$,则 $b$ 是 $a$ 的像,$a$ 是 $b$ 的原像。
\end{definition}

\begin{definition}[定义域、陪域与值域]\label{def:定义域、陪域与值域}
    令 $f: A \to B$,则 $A$ 被称为函数的定义域,$B$ 被称为函数的陪域,$A$ 中所有元素像的集合被称为值域,值域总是陪域的子集。
\end{definition}