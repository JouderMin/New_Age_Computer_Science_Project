\documentclass{script}

\title{\textbf{Chp.0 序章}}
\author{Jouder Min}
\date{\today}

\begin{document}
\maketitle
\setlength{\parskip}{1em}
\begin{description}[itemsep=1ex,leftmargin=0.92cm]
    \item[凌玲] 各位观众佬爷们大家好!
    \item[凌伊] 大家好。
    \item[凌玲] 我是凌玲。
    \item[凌伊] 我是凌伊。
    \item[凌玲] 欢迎大家观看我们制作的第一个企划,新世代计算机科学计划!现在你看到的是本企划的第一篇章——基础篇哦!
    \item[凌伊] 这个企划的目的是为想要学习计算机科学却不知如何下手的小伙伴们打造的。根据姐姐的估计,掌握了基础篇的内容,你就差不多有了和大学本科计算机科学专业大一同学一样的水准。说是说计算机科学,但看起来姐姐似乎完全不想在基础篇讲编程呢。
    \item[凌玲] 啊这,确实。({\textsf{咳嗽声}})总之,在基础篇我将为大家讲解计算机科学的基础,离散数学和计算机组成原理将会是本篇章的主要内容。
    \item[凌伊] ({\textsf{打断}})人被逼急了什么都做得出,除了数学。
    \item[凌玲] 你故意找茬是不是?你学不学吧?
    \item[] ({\textsf{家暴中。。。}})
    \item[凌玲] 总之,离散数学和计算机组成原理是计算机科学中最最基础的部分了,在没有学会这些基础内容的情况下直接学习编程反而会越学越懵。我们的目标是让所有人快乐学习,而不是感受痛苦,因此我们将尽可能用简单的语言来讲述这些知识。
    \item[凌伊] ({\textsf{鼻青脸肿}})好耶!
    \item[凌玲] 那么下一个视频就是基础篇的第一章——离散数学了。
\end{description}
\end{document}